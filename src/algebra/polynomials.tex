\chapter{Polynomials}
\label{chap:P}
Further reading: \texttt{Chen \& Duong ``mtfym02.pdf''}, \url{http://goo.gl/jHCXo}

\begin{bigideas}{sec:P Big Ideas}
\begin{itemize}
  \item An expression of finite length constructed from variables and constants
        using only addition, subtraction, multiplication and non-negative integer
        powers.
  \item The highest power of a variable indicates the polynomials \emph{degree}.
  \item Determining the \emph{roots} of a poloynomial tells us where a
        polynomial equation intersects the $x$-axis.
  \item Polynomials of degree $2$ are called \emph{quadratic}.
  \item Polynomials of degree $3$ are called \emph{cubic}.
\end{itemize}
\end{bigideas}

\section{Polynomial Equations}
\label{sec:P Polynomial Equations}
Further reading: \texttt{Chen \& Duong ``mtfym02.pdf'' pp 3-4.}

An equation in the form $ax + b = 0$ where $a,b \in \mathbb{F}$ and $a \neq 0$
is called a \emph{linear equation} or \emph{linear polynomial equation}. It has
a degree of $1$ (that is, the power of $x$ is $1$). We can determine the value
of $x$:

\begin{align}
  ax + b &= 0 \\
  ax &= -b \\
  x &= -\frac{b}{x}
\end{align}

An equation in the form $ax^2 + bx +c = 0$ where $a, b, c \in \mathbb{F}$ are
constants and $a \neq 0$ is called a \emph{quadratic equation}. To solve such an
equation we use the \emph{quadratic formula}:

\begin{align}
  x &= \frac{-b \pm \sqrt{b^2 -4ac}}{2a} \label{eq:quadraticFormula}
\end{align}

Where $b^2-4ac$ (called the \emph{determinant}) is greater than or equal to $0$.
This will yield two solutions:
\begin{align}
  x &= \frac{-b + \sqrt{b^2 -4ac}}{2a} \nonumber
\intertext{and}
  x &= \frac{-b - \sqrt{b^2 -4ac}}{2a} \nonumber
\end{align}
However, equation \ref{eq:quadraticFormula} is the most compact and easiest to
remember.

If the determinant < 0, then it indicates there are no roots exist in the
real plane ($\mathbb{R}$) and that a complex plane ($\mathbb{C}$) solution
exists.

For polynomials of degree $3$ or more there is no general formula for a
solution, however division of polynomials allows us to reduce polynomials of a
higher degree to a factor and a lesser degree. By iteratively reducing
polynomials we can eventually find all roots. 
\section{Polynomial Division}
\label{sec:P Polynomial Division}
Further reading: \texttt{Chen \& Duong ``mtfym02.pdf'' pp 4-6.}

Remember long division from primary school? By the time we get to university
most of us have forgotten. Checkout \url{http://goo.gl/TvDoI} for a Kahn Academy
video on how to do it using just plain numbers. It is well worth the refresher
even if you think you can remember how it works.

There is a simple recursive algorithm for polynomial long division, consisting
of 3 steps which are repeated until either a divide by zero, or a remainder
occurs.\footnote{I call this algorithm Jaye's Algorithm because Jaye, a 9 year
old at the time, helped me memorise it a few weeks before writing it down.}

\begin{remember}{Jaye's Algorithm}
\begin{enumerate}
  \item Divide the next term numerator (the part under the long line) by the
  first term in the denominator. Write the result above the long line.
  \item Multiply the result by the denominator and then by $-1$. Write the
  result underneath the term(s) operated on, underlining any new terms.
  \item Perform a simple addition, carrying anything down below the line.
\end{enumerate}
Rinse and repeat these 3 steps underneath your new long line.
\end{remember}

The following example comes from Purple Math, \url{http://goo.gl/GPi0z}:
Suppose we wish to divide $\frac{x^2 -9x -10}{x+1}$:
\begin{enumerate}
  \item First write the polynomial in the form:
  \\ \polylongdiv[stage=1]{x^2 -9x -10}{x+1} \\[0.5cm]
  We will look only at the leading $x$ in the divisor and the leading $x^2$ in
  the numerator. 
  \item We want to divide the $x^2$ in the numerator by the $x$ in the
  denominator. $\frac{x^2}{x} = x$ so we write the resulting $x$ above the line.
  \\ \polylongdiv[stage=2]{x^2 -9x -10}{x+1}
  \item Now we must take care of the $+1$ in the denominator. Multiply the
  result above the line by $-1$ and multiply again by the denominator:
  \\ \polylongdiv[stage=3]{x^2 -9x -10}{x+1}
  \item Perform the subtraction as we would a normal sum carrying anything to
  the right down:
  \\ \polylongdiv[stage=4]{x^2 -9x -10}{x+1}
  \item Divide $-10x -10$ by $x$ goes $-10$ times, so write $-10$ above the line
  \\ \polylongdiv[stage=5]{x^2 -9x -10}{x+1} 
  \item Multiply $-10$ by $-1$ and then again by $(x+1)$:
  \\ \polylongdiv[stage=7]{x^2 -9x -10}{x+1}
  \item \textbf{The zero means there is no remainder, and we are done.}
\end{enumerate}

Another example of polynomial long division, taken from the same site as above,
to divide $\frac{x^2 +9x +14}{x +7}$
\begin{enumerate}
  \item Write out in the standard form:
  \\ \polylongdiv[stage=1]{x^2 +9x +14}{x +7} \\
  \item Next divide $\frac{x^2}{x}=x$ and write above the line.
  \\ \polylongdiv[stage=2]{x^2 +9x +14}{x +7} \\
  \item Multiply $-x$ by $x+7$ and write underneath:
  \\ \polylongdiv[stage=3]{x^2 +9x +14}{x +7} \\
  \item Perform the addition:
  \\ \polylongdiv[stage=4]{x^2 +9x +14}{x +7} \\
  \item Divide $\frac{2x}{x}=2$
  \\ \polylongdiv[stage=5]{x^2 +9x +14}{x +7} \\
  \item Multiply $-2$ by $x+7$ and write underneath, and we can see there is no
  remainder.
  \\ \polylongdiv[stage=7]{x^2 +9x +14}{x +7}
\end{enumerate}

Sometimes there are tricky ones where you might have some terms with a
coefficient of zero, such as $\frac{2x^3 -9x^2 +15}{2x -5}$. Here $x^1$ does not
appear as a part of numerator, but we can rewrite it as
$\frac{2x^3 -9x^2 +0x +15}{2x -5}$. With problems such as these, it is easiest
to keep your writing neat and in columns. Where you might see $0x$ you can just
as easily substitute a bit of whitespace. \\[0.5cm]
\polylongdiv{2x^3 -9x^2 +15}{2x -5}
This example also shows a remainder of $-10$. The final answer could also be
written as one of two ways, either:
\begin{enumerate}
  \item $x^2 -2x -5 - \left(\frac{10}{2x -5}\right)$ -- or --
  \item $x^2 -2x -5$ r $10$
\end{enumerate}
Though the first method is probably better as it is less confusing sans the
letter ``r''.

\section{Polynomial Roots}
\label{sec:P Polynomial Root}
Further reading: \texttt{Chen \& Duong ``mtfym02.pdf'' pp 6-8.}

\section{Fundamental Theorem of Algebra}
\label{sec:P Fundamental Theorem of Algebra}
Further reading: \texttt{Chen \& Duong ``mtfym02.pdf'' pp 7.}

\section{Roots of Real Polynomials}
\label{sec:P Roots of Real Polynomial}
Further reading: \texttt{Chen \& Duong ``mtfym02.pdf'' pp 7-8.}

\section{More Polynomial Roots}
\label{sec:P More Polynomial Root}
Further reading: \texttt{Chen \& Duong ``mtfym02.pdf'' pp 8.}

\section{Rational Functions}
\label{sec:P Rational Functions}
Further reading: \texttt{Chen \& Duong ``mtfym02.pdf'' pp 8-11.}

\section{Greatest Common Divisor}
\label{sec:P Greatest Common Divisor}
Further reading: \texttt{Chen \& Duong ``mtfym02.pdf'' pp 12-13.} \\
Further reading: \texttt{Kenneth H ROSEN ``Discrete Mathematics and Its
Applications 6e'', Chapter Primes and Greatest Common Divisors, pp 210-227.}
\begin{remember}{Definition}
  Let $a$ and $b$ be integers no both zero. The largest integer $d$ such that
  $d | a$\footnote{read as $d$ divides $a$} and $d | b$ is called the greatest
  common divisor of $a$ and $b$, denoted as gcd($a$,$b$).
\end{remember}

There are many ways to determine the gcd of two numbers, including a products of
primes and Euclid's algorithm. Euclid's algorithm is by far the most efficient
and is discussed in the next subsection.

\subsection{Euclid's Algorithm}
\label{sec:P Euclid's Algortihm}
Further reading: \texttt{Chen \& Duong ``mtfym02.pdf'' pp 12-13.}\\
Further reading: \texttt{Rosen ``Discrete Mathematics and Its Applications'', 6e, pp 227-229}\\[0.5cm]

Euclid's algorithm is recursive, and consists of $2$ steps:
\begin{remember}{Euclid's algorithm}{
\begin{enumerate}
  \item Divide the larger of the two numbers by the smaller of the two numbers,
  and add the remainder onto the end.
  \item Find the gcd of the remainder and the smaller of the two original numbers.
\end{enumerate}

In summary:
\begin{align}
  \text{Let} ~ a &= bq + r \quad | \quad a,b,q,r \in \mathbb{Z} \quad \text{Then gcd($a$,$b$)} = \text{gcd($b$,$r$)}
\end{align}
}
\end{remember}

The following example comes from ROSEN p 229:

\begin{align}
  662 &= 414 \cdot 1 + 248 \\
  414 &= 248 \cdot 1 + 166 \\
  248 &= 166 \cdot 1 + 82 \\
  166 &= 82 \cdot 2 + 2 \\
  82 &= 2 \cdot 41
\end{align}
2 is the last non-zero remainder, therefore gcd(414,662) = 2.