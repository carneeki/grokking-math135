\chapter{Polynomials}
\label{chap:P}
Further reading: \texttt{Chen \& Duong ``mtfym02.pdf''}, \url{http://goo.gl/jHCXo}

\begin{bigideas}{sec:P Big Ideas}
\begin{itemize}
  \item An expression of finite length constructed from variables and constants
        using only addition, subtraction, multiplication and non-negative integer
        powers.
  \item The highest power of a variable indicates the polynomials \emph{degree}.
  \item Determining the \emph{roots} of a poloynomial tells us where a
        polynomial equation intersects the $x$-axis.
  \item Polynomials of degree $2$ are called \emph{quadratic}.
  \item Polynomials of degree $3$ are called \emph{cubic}.
\end{itemize}
\end{bigideas}

\section{Polynomial Equations}
\label{sec:P Polynomial Equations}
Further reading: \texttt{Chen \& Duong ``mtfym02.pdf'' pp 3-4.}

An equation in the form $ax + b = 0$ where $a,b \in \mathbb{F}$ and $a \neq 0$
is called a \emph{linear equation} or \emph{linear polynomial equation}. It has
a degree of $1$ (that is, the power of $x$ is $1$). We can determine the value
of $x$:

\begin{align}
  ax + b &= 0 \\
  ax &= -b \\
  x &= -\frac{b}{x}
\end{align}

An equation in the form $ax^2 + bx +c = 0$ where $a, b, c \in \mathbb{F}$ are
constants and $a \neq 0$ is called a \emph{quadratic equation}. To solve such an
equation we use the \emph{quadratic formula}:

\begin{align}
  x &= \frac{-b \pm \sqrt{b^2 -4ac}}{2a} \label{eq:quadraticFormula}
\end{align}

Where $b^2-4ac$ (called the \emph{determinant}) is greater than or equal to $0$.
This will yield two solutions:
\begin{align}
  x &= \frac{-b + \sqrt{b^2 -4ac}}{2a} \nonumber
\intertext{and}
  x &= \frac{-b - \sqrt{b^2 -4ac}}{2a} \nonumber
\end{align}
However, equation \ref{eq:quadraticFormula} is the most compact and easiest to
remember.

If the determinant < 0, then it indicates there are no roots exist in the
real plane ($\mathbb{R}$) and that a complex plane ($\mathbb{C}$) solution
exists.

For polynomials of degree $3$ or more there is no general formula for a
solution, however division of polynomials allows us to reduce polynomials of a
higher degree to a factor and a lesser degree. By iteratively reducing
polynomials we can eventually find all roots. 
\section{Polynomial Division}
\label{sec:P Polynomial Division}
Further reading: \texttt{Chen \& Duong ``mtfym02.pdf'' pp 4-6.}

Remember long division from primary school? By the time we get to university
most of us have forgotten. Checkout \url{http://goo.gl/TvDoI} for a Kahn Academy
video on how to do it using just plain numbers. It is well worth the refresher
even if you think you can remember how it works.

The following example comes from Purple Math, \url{http://goo.gl/GPi0z}:
Suppose we wish to divide $\frac{x^2 -9x -10}{x+1}$ 
\begin{enumerate}
  \item First write the polynomial in the form:
  \\ \polylongdiv[stage=1]{x^2 -9x -10}{x+1} \\
  We will look only at the leading $x$ in the divisor and the leading $x^2$ in
  the numerator. 
  \item We want to divide the $x^2$ in the numerator by the $x$ in the denominator:
  \\ \polylongdiv[stage=2]{x^2 -9x -10}{x+1} \\
  $\frac{x^2}{x} = x$ so we write the resulting $x$ above the line. 
  \item Now we have to take care of the $+1$ in the denominator. Using the $x$
  from our result above the line, we multiply the result by the denominator
  $x+1$
  \\ \polylongdiv[stage=3]{x^2 -9x -10}{x+1} \\
\end{enumerate}

\section{Polynomial Roots}
\label{sec:P Polynomial Root}
Further reading: \texttt{Chen \& Duong ``mtfym02.pdf'' pp 6-8.}

\section{Fundamental Theorem of Algebra}
\label{sec:P Fundamental Theorem of Algebra}
Further reading: \texttt{Chen \& Duong ``mtfym02.pdf'' pp 7.}

\section{Roots of Real Polynomials}
\label{sec:P Roots of Real Polynomial}
Further reading: \texttt{Chen \& Duong ``mtfym02.pdf'' pp 7-8.}

\section{More Polynomial Roots}
\label{sec:P More Polynomial Root}
Further reading: \texttt{Chen \& Duong ``mtfym02.pdf'' pp 8.}

\section{Rational Functions}
\label{sec:P Rational Functions}
Further reading: \texttt{Chen \& Duong ``mtfym02.pdf'' pp 8-11.}

\section{Greatest Common Divisor}
\label{sec:P Greatest Common Divisor}
Further reading: \texttt{Chen \& Duong ``mtfym02.pdf'' pp 12-13.}

\subsection{Euclid's Algorithm}
\label{sec:P Euclid's Algortihm}
Further reading: \texttt{Chen \& Duong ``mtfym02.pdf'' pp 12-13.}\\
Further reading: \texttt{Rosen ``Discrete Mathematics and Its Applications'', 6e, pp 227-229}