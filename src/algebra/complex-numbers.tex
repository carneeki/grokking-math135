\chapter{Complex Numbers}
\label{chap:CN}
Further reading: \texttt{Chen \& Duong ``em09-cn.pdf''}, \, \url{http://goo.gl/C5rfq} \\
Further reading: \texttt{Chen \& Duong ``fyc01.pdf''}, \, \url{http://goo.gl/eOE7y} \\
Further reading: \texttt{Chris Cooper ``LINALG03 Complex Numbers''}, \, \url{http://goo.gl/N85uO}
\begin{bigideas}{sec:CN Big Ideas}
\begin{itemize}
  \item Complex numbers are created to solve what would otherwise be unsolveable
  equations because we cannot square any number in $\mathbb{R}$ to get a
  negative number.
  \item Complex numbers are made of two parts, a real part and an imaginary part
  (i).
  \item i exhibits ring-like behaviour when raised to powers.
  \item A complex number resembles the form:\\
  \begin{align}
    z & = x + y\text{i} \quad | ~ x,y \in \mathbb{R} 
  \intertext{
  and the set of complex numbers, $\mathbb{C}$ is defined as:
  }
    \mathbb{C} &= {a + ib \quad | ~ a,b \in \mathbb{R}, ~ i^2 =-1}
  \end{align}
  \item Rules exist for adding, subtracting, multiplying and dividing
  complex numbers.
  \item Complex numbers have a ``twin'' called a \emph{complex conjugate},
  suppose we have:
  \begin{align}
    z = (x + y\text{i}) \nonumber
    \intertext{then the conjugate is:}
    \overline{z} = (x - y\text{i}) \nonumber
  \end{align}
\end{itemize}
\end{bigideas}

When we square any number in the real number plane, we \emph{always} get a
postive number. This begs the question, how do we square root a negative number?
Previously, a problem such as
\begin{align}
  i^2 + 1 &= 0 \\
  i^2     &= -1 \label{eq:i2}
\intertext{Would be left with ``no real solution'' as an answer in a paper.
While this is true, we \textbf{can in fact find a solution}. It just happens to
be outside of the real plane and is defined exactly as:}
  i &= \sqrt{-1}
\end{align}
\begin{align}
\intertext{This gives rise to a very interesting property\ldots What if we were to square
i? We already know, the answer from equation \ref{eq:i2}}
  i^2 &= -1
\intertext{What about if we cube i?}
  i^3 &= i * i^2 \\
      &= i * -1 \\
      &= -i \label{eq:i3}
\intertext{What if we go again?}
  i^4 &= i^2 * i^2 \\
      &= -1 * -1 \\
      &= 1 \\
\intertext{And again?}
  i^5 &= i * i^4 \\
      &= i * 1 \\
      &= i
\end{align}
This repetitive behaviour is called a \emph{ring} and is cyclic for every $4^{th}$
power of $i$. Remember this fact, it will become useful in later trigonometry.

\begin{remember}{Ring behaviour of powers of $i$}
\begin{align}
  i^0 &= 1 \\
  i^1 &= i \\
  i^2 &= -1 \\
  i^3 &= -i \\
  i^4 &= 1 \\
  i^5 &= i \\
  i^6 &= -1 \\
  i^7 &= -i \\
  i^8 &= 1
\end{align}
\end{remember}
Why does this work so neatly? Consider our basic index laws and apply to numbers
between 0 and 3 inclusive:
\begin{align}
  i^{a + b} &= i^a * i^b \quad | \quad a,b \in [0,3]
\end{align}

\section{Arithmetic of Complex Numbers}
\label{sec:CN Arithmetic of Complex Numbers}
Further reading: \texttt{Chen \& Duong ``em09-cn.pdf'' pp 1-2.} \\
Further reading: \texttt{Chen \& Duong ``fyc01.pdf'' pp 6-8.}

\begin{remember}{Addition Rule}
\begin{align}
  (a + b\text{i}) + (c + d\text{i}) &= (a + c) + (b + d)\text{i}
\end{align}
\end{remember}

\begin{remember}{Subtraction Rule}
\begin{align}
  (a + b\text{i}) - (c + d\text{i}) &= (a - c) + (b - d)\text{i}
\end{align}
\end{remember}

\begin{remember}{Multiplication Rule}
\begin{align}
  (a + b\text{i}) \cdot (c + d\text{i}) &= (ac - bd) + (ad + bc)\text{i}
\end{align}
\end{remember}

\begin{remember}{Division Rule}
\begin{align}
  \frac{(a + b\text{i})}{(c + d\text{i})} &= x + y\text{i} \quad | \quad x,y \in \mathbb{R}
  \intertext{such that}
  a + b\text{i} = (c + d\text{i})(x+y\text{i}) & = (cx - dy)(cy+dx)\text{i}
  \intertext{It follows that}
  a &= cx - dy \\
  b &= cy + dx
  \intertext{With the solution:}
  x = \frac{ac + bd}{c^2 + d^2} & \quad \text{and} \quad y = \frac{bc -ad}{c^2+d^2}
\end{align}
\end{remember}

\section{Complex Conjugates}

Complex numbers have a ``twin'' called a \emph{complex conjugate}, suppose we
have:
\begin{align}
  z = (x + y\text{i}) \nonumber
  \intertext{then the conjugate is:}
  \overline{z} = (x - y\text{i}) \nonumber
\end{align}

Conjugates exhibit special properties, they are:
\begin{align}
  \overline{z + w} &= \overline{z} + \overline{w} \label{eq:complexconjugadd}\\
  \intertext{and}
  \overline{z\cdot w} &= \overline{z} \cdot \overline{w} \label{eq:complexconjugmult}
\end{align}

Proof of equation \ref{eq:complexconjugadd} is as follows:
\begin{align}
  \text{Let:}~z &= x + y\text{i} \quad | \quad x,y \in \mathbb{R} \\
  \text{Let:}~w &= u + v\text{i} \quad | \quad u,v \in \mathbb{R} \\
  \overline{z + w} &= \overline{(x+u)+(y+v)\text{i}} \\
    &= (x+u)-(y+v)\text{i} \\
    &= (x-y\text{i}) + (u-v\text{i}) \\
    &= \overline{z} + \overline{w}
\end{align}\qedbitches

Proof of equation \ref{eq:complexconjugmult} is as follows:
\begin{align}
  \text{Let:}~z &= x + y\text{i} \quad | \quad x,y \in \mathbb{R} \\
  \text{Let:}~w &= u + v\text{i} \quad | \quad u,v \in \mathbb{R} \\
  \overline{z \cdot w} &= \overline{(x+y\text{i})(u+v\text{i})} \\
    &= \overline{(xu-yv)+(xv+yu)\text{i}}\\
    &= (xu-yv)-(xv+yu)\text{i}\\
    &= (x - y\text{i})(u-v\text{i}) \\
    &= \overline{z} \cdot \overline{w}
\end{align}\qedbitches

\section{Polynomials with Complex Coefficients}
Every polynomial with complex coefficients has a complex root. This result is
called the \emph{Fundamental Theorem of Algebra}.\footnote{Unfortunately the
proof of this is rather complicated.}

\section{Equalities of Complex Numbers}
In developing new numbers, a property is lost. \\ For example $\mathbb{N} \to
\mathbb{Z}$, we lose the well ordering property.
As we go from $\mathbb{R} \to \mathbb{C}$ we lose the property of equalities
(such as $<$ and $>$).\footnote{cf Apostol, Mathematical Analysis}.

Suppose that $<$ has usual properties and also suppose that $i > 0$. We need to
choose one of $i > 0$, or $i < 0$, or $i = 0$. Since $i > 0$:
\begin{align}
      i & > 0 \\
  i * i & > i*0 \\
  -1 & > 0 \quad \text{problem!} \\
  -i & > 0 \quad \text{problem, because} \\
   i & < 0 \quad \text{How can i be bigger and less than zero at the same time?}
\end{align}

Recap: How does equality work?
\begin{align}
  \intertext{Let}
    z & = a + ib \\
    w & = c + id \\
  \intertext{then}
    z & = w \quad \text{iff } a = c \text{and} b = d
\end{align}

We have don't go outside $\mathbb{C}$, let's consider, $\sqrt{i}$. Is this
complex? Consider the question, ``What polynomial is this the root of?''.
\begin{align}
  \intertext{Let}
    x = \sqrt{i} \\
    x^2 = i \\
    x^2 - i = 0 \\
  \intertext{such that}
  \sqrt{i} \quad \text{is a root of} \\
  p(x) = x^2 - i \\
  \intertext{By the Fundamental Theorem of Algebra, we must be able to find $a,b$
  so that a,b $\in \mathbb{R}$} \\
  a + ib = \sqrt{i} \\
\end{align}

\begin{align}
  a + ib = \sqrt{i} \\
  (a + ib)(a + ib) = i \\
  a^2 + 2iab + i^2(b^2) = i \\
  a^2 + 2iab - b^2 = i
\end{align}

$i$ has great practical importance. Without $i$ we can't do much engineering
or physics without $i$. Mainly electronics engineers already have $i$ reserved
for the unit of current, so $j$ is used in place.\footnote{Because engineers are
even less creative than mathematicians.}

We have already seen how to draw $i$.

%      |, 1
%      |
% -----+-----
%      |
%      |
%

What of 2+i?

Identify a + ib with (a,b).
$a + ib \in \mathbb{C} | (a,b) \in \mathbb{R}^2$

$a + ib \leftrightarrow (a,b)$

\section{Scaling complex numbers}
Approach from a numerical example:
\begin{align}
  \intertext{Let}
    z &= 3 + i \\
  \intertext{Consider}
    2z &= 6 + 2i \\
    3z &= 9 + 3i \\
    \frac{1}{2}z = \frac{3}{2} + \frac{1}{2}i \\
    -z &= -3 -i \\
\end{align}

A very important concept in complex numbers is the distance away from $0$. As
for reals, we denote this by absolute value. This is also called the \emph{modulus}
or \emph{magnitude}.

$$|z| = \text{is the distance from $z$ to $0$}$$
