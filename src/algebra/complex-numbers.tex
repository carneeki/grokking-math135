\chapter{Complex Numbers \& The Complex Plane}
\label{chap:CN}
Further reading: \texttt{Chen \& Duong ``em09-cn.pdf''}, \url{http://goo.gl/C5rfq} \\
Further reading: \texttt{Chen \& Duong ``fyc01.pdf''}, \url{http://goo.gl/eOE7y} \\
Further reading: \texttt{Chris Cooper ``LINALG03 Complex Numbers''}, \url{http://goo.gl/N85uO}

\begin{bigideas}{sec:CN Big Ideas}
\begin{itemize}
  \item Complex numbers are created to solve what would otherwise be unsolveable
  equations because we cannot square any number in $\mathbb{R}$ to get a
  negative number.
  \item Complex numbers are made of two parts, a real part and an imaginary part
  (i) which, when combined are called a \emph{complex conjugate}.
  \item A complex conjugate resembles the form:\\
  \begin{align}
    a + b\text{i} & \quad | \quad a,b \in \mathbb{R} 
  \end{align}
  \item 4 simple rules exist for adding, subtracting, multiplying and dividing
  complex numbers.
  \item Complex numbers fall into their own special set of numbers $\mathbb{C}$.
\end{itemize}
\end{bigideas}

When we square any number in the real number plane, we \emph{always} get a
postive number. This begs the question, how do we square root a negative number?
Previously, a problem such as
\begin{align}
  i^2 + 1 &= 0 \\
  i^2     &= -1 \label{eq:i2}
\intertext{Would be left with ``no real solution'' as an answer in a paper.
While this is true, we \textbf{can in fact find a solution}. It just happens to
be outside of the real plane and is defined exactly as:}
  i &= \sqrt{-1}
\end{align}
\begin{align}
\intertext{This gives rise to a very interesting property\ldots What if we were to square
i? We already know, the answer from equation \ref{eq:i2}}
  i^2 &= -1
\intertext{What about if we cube i?}
  i^3 &= i * i^2 \\
      &= i * -1 \\
      &= -i \label{eq:i3}
\intertext{What if we go again?}
  i^4 &= i^2 * i^2 \\
      &= -1 * -1 \\
      &= 1 \\
\intertext{And again?}
  i^5 &= i * i^4 \\
      &= i * 1 \\
      &= i
\end{align}
This repetitive behaviour is called a \emph{ring} and is cyclic for every $4^{th}$
power of $i$. Remember this fact, it will become useful in later trigonometry.

\begin{remember}{Ring behaviour of powers of $i$}
\begin{align}
  i^0 &= 1 \\
  i^1 &= i \\
  i^2 &= -1 \\
  i^3 &= -i \\
  i^4 &= 1 \\
  i^5 &= i \\
  i^6 &= -1 \\
  i^7 &= -i \\
  i^8 &= 1
\end{align}
\end{remember}
Why does this work so neatly? Consider our basic index laws and apply to numbers
between 0 and 3 inclusive:
\begin{align}
  i^{a + b} &= i^a * i^b \quad | \quad a,b \in [0,3]
\end{align}

\section{Arithmetic of Complex Numbers}
\label{sec:CN Arithmetic of Complex Numbers}
Further reading: \texttt{Chen \& Duong ``em09-cn.pdf'' pp 1-2.} \\
Further reading: \texttt{Chen \& Duong ``fyc01.pdf'' pp 6-8.}

\begin{remember}{Addition Rule}
\begin{align}
  (a + b\text{i}) + (c + d\text{i}) &= (a + c) + (b + d)\text{i}
\end{align}
\end{remember}

\begin{remember}{Subtraction Rule}
\begin{align}
  (a + b\text{i}) - (c + d\text{i}) &= (a - c) + (b - d)\text{i}
\end{align}
\end{remember}

\begin{remember}{Multiplication Rule}
\begin{align}
  (a + b\text{i}) \cdot (c + d\text{i}) &= (ac - bd) + (ad + bc)\text{i}
\end{align}
\end{remember}

\begin{remember}{Division Rule}
\begin{align}
  \frac{(a + b\text{i})}{(c + d\text{i})} &= x + y\text{i} \quad | \quad x,y \in \mathbb{R}
  \intertext{such that}
  a + b\text{i} = (c + d\text{i})(x+y\text{i}) & = (cx - dy)(cy+dx)\text{i}
  \intertext{It follows that}
  a &= cx - dy \\
  b &= cy + dx
  \intertext{With the solution:}
  x = \frac{ac + bd}{c^2 + d^2} & \quad \text{and} \quad y = \frac{bc -ad}{c^2+d^2}
\end{align}
\end{remember}
