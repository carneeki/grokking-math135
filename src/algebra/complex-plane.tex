\chapter{Complex Plane}
\label{chap:CP}
Further reading: \texttt{Chen \& Duong ``em09-cn.pdf'' pp 3-4.} \\
Further reading: \texttt{Chen \& Duong ``fyc01.pdf'' pp 8-11.}

\begin{bigideas}{sec:CP Big Ideas}
\begin{itemize}
  \item Complex numbers take the form $(a + b\text{i})$.
  \item We can represent $x$ and $y$ ordinates in the real plane ($\mathbb{R}^2$)
  in the form $(x + y\text{i})$.
  \item $x$ is called the real part (represented by $\Re[z]$)
  \item $y$ is called the imaginary part (represented by $\Im[z]$)
  \item Plotting these on an $x,y$ axis is called an \emph{Argand diagram}.
  \item Polar coordinates use two variables, $(r,\theta)$ to describe a point.
  \item $\theta$ is an angle measured in radians
  \item $r$ is a length.
  \item Conversion between polar coordinates and Cartesian coordinates is done
  using Pythagoras theorem or trigonmetry.
\end{itemize}
\end{bigideas}

\section{Argand Diagrams}
\label{sec:CP Argand Diagrams}
Complex numbers consist of two parts, such as $z = (x + y\text{i})$. Suppose we
were to plot both $x$ and $y$ ordinates on an $x,y$ plane as seen in figure
\ref{fig:argand}.
\begin{figure}[!htb]
\begin{center}
\begin{tikzpicture}
  \draw[very thin, color=black!10] (-1.5, -2.5) grid (6.5, 2.5);
  \draw[<->] (-1.5,0) -- (6.5,0);
  \draw[<->] (0,-2.5) -- (0,2.5);
  
  \draw[dashed,color=nicered] (3,1) -- (3,0);
  \draw[dashed,color=nicered] (-0.5,1) -- (3,1);
  \draw[color=nicered,fill=nicered] (3,1) circle(0.7mm);
  \draw node at (4,0.5) {$z = x + y\text{i}$};
  \draw node at (2.8,-0.25) {$x$};
  \draw node at (-0.25,0.5) {$y$};
  
  \draw[dashed,color=neekiBlue] (3,-1) -- (3,0);
  \draw[dashed,color=neekiBlue] (-0.5,-1) -- (3,-1);
  \draw[color=neekiBlue,fill=neekiBlue] (3,-1) circle(0.7mm);
  \draw node at (4,-0.5) {$\overline{z} = x - y\text{i}$};
  \draw node at (-0.25,-0.5) {$-y$};
  
\end{tikzpicture}
\end{center}
\caption{Argand Diagram, $z$, and complex conjugate $\overline{z}$.}
\label{fig:argand}
\end{figure}

Following our addition rules in section \ref{sec:CN Arithmetic of Complex
Numbers} (page \pageref{sec:CN Arithmetic of Complex Numbers}), we can add, two
complex numbers as seen in figure \ref{fig:argandaddition}.
\begin{figure}[!htb]
\begin{center}
\begin{tikzpicture}
  \draw[very thin, color=black!10] (-1.5, -1.5) grid (6.5, 6.5);
  \draw[<->] (-1.5,0) -- (6.5,0);
  \draw[<->] (0,-1.5) -- (0,6.5);
  
  \draw[dashed,color=nicered] (3,1) -- (3,-0.5);
  \draw[dashed,color=nicered] (-0.5,1) -- (3,1);
  \draw[color=nicered,fill=nicered] (3,1) circle(0.7mm);
  \draw node at (4,0.5) {$z = x + y\text{i}$};
  \draw node at (2.8,-0.5) {$x$};
  \draw node at (-0.5,0.5) {$y$};
  
  \draw[dashed,color=neekiBlue] (1,3) -- (1, -0.5);
  \draw[dashed,color=neekiBlue] (-0.5,3) -- (1, 3);
  \draw[color=neekiBlue,fill=neekiBlue] (1,3) circle(0.7mm);
  \draw node at (2,2.5) {$w = u + v\text{i}$};
  \draw node at (0.8,-0.25) {$u$};
  \draw node at (-0.25,2.5) {$v$};
  
  \draw[dashed,color=neekiGreen] (5,5) -- (5, -0.5);
  \draw[dashed,color=neekiGreen] (-0.5,5) -- (5, 5);
  \draw[color=neekiGreen,fill=neekiGreen] (5,5) circle(0.7mm);
  \draw node at (5.5,4.5) {$z + w$};
  \draw node at (4.5,-0.25) {$x+u$};
  \draw node at (-0.5,4.5) {$y+v$};
  
  \draw[dashed,color=neekiBlue!50] (5,5) -- (1, 3);
  \draw[dashed,color=neekiBlue!50] (0,0) -- (1, 3);
  \draw[dashed,color=nicered!50] (5,5) -- (3, 1);
  \draw[dashed,color=nicered!50] (0,0) -- (3, 1);
  \draw[dashed,color=neekiGreen!50] (5,5) -- (0, 0);
\end{tikzpicture}
\end{center}
\caption{Addition of two complex numbers in an Argand Diagram}
\label{fig:argandaddition}
\end{figure}

\noindent The working for \ref{fig:argandaddition} is as follows:
\begin{align}
  \text{Let:}~ z &= x + y\text{i} \\
  \text{Let:}~ w &= u + v\text{i}
  \intertext{By the addition rule:}
  (x + y\text{i}) + (u + v\text{i}) &= x + u + y\text{i} + v\text{i} \\
                                    &= (x + u) + (y + v)\text{i}
\end{align}

\section{Polar Coordinates}
\label{sec:CP Polar Coordinates}
\begin{enumerate}
  \item Polar coordinates use two variables, $(r,\theta)$ to describe a point.
  \item $r$ is a length.
  \item $\theta$ is an angle
\end{enumerate}

\begin{figure}[!htb]
\begin{center}
\begin{tikzpicture}
  \draw[very thin, color=black!10] (-3.5, -4.5) grid (2.5, 2.5);
  \coordinate (o) at (0,0);
  \draw[color=black,fill=black] (o) circle(0.7mm);
  \draw[->] (o) -- (2.5,0);
  
  \draw[dashed,color=nicered] (o)+(1.5,0) arc (0:60:1.5);
  \draw[color=nicered] (o) -- ++(60:2);
  \draw[color=nicered,fill=nicered] ++(60:2) circle(0.7mm)
      node[right] {$P = (r,\theta)$};
  
  \draw[dashed,color=neekiBlue] (o)+(1,0) arc (0:240:1);
  \draw[color=neekiBlue] (o) -- ++(240:4);
  \draw[color=neekiBlue,fill=neekiBlue] ++(240:4) circle(0.7mm)
      node[right] {$Q = (2r,4\theta)$};
\end{tikzpicture}
\end{center}
\caption{Two polar coordinates. Note how $r$, the radius, is specified first
followed by the angle.}
\label{fig:polarcoordinates}
\end{figure}

\begin{align}
\intertext{Converting between polar coordinates and cartesian coordinates is a
matter of applying some trigonometry but is summarised as follows:}
  x &= r\cos\theta \\
  y &= r\sin\theta
\intertext{Iff $r \geq 0$ and $-\pi \leq \theta < \pi$ we can also convert to
Cartesian coordinates as follows:}
  r &= \sqrt{x^2 + y^2} \\
  \theta &= \left \{
    \begin{array}{l l}
      \arctan\left(\frac{y}{x}\right) & \quad \text{if $x > 0$} \\
      \arctan\left(\frac{x}{y}\right) + \pi & \quad \text{if $x < 0$ and $y \geq 0$} \\
      \arctan\left(\frac{y}{x}\right) - \pi & \quad \text{if $x < 0$ and $y < 0$} \\
       \frac{\pi}{2} & \quad \text{if $x = 0$ and $y > 0$} \\
      -\frac{\pi}{2} & \quad \text{if $x = 0$ and $y < 0$} \\
      0              & \quad \text{if $x = 0$ and $y = 0$} \\
    \end{array}
    \right.
\end{align}

\section{Modulus}
\label{sec:CP Modulus}

Consider diagram \ref{fig:modulus}:

\begin{figure}[!htb]
\begin{center}
\begin{tikzpicture}
  \draw[very thin, color=black!10] (-1.5, -0.5) grid (6.5, 3.5);
  \draw[<->] (-0.5,0) -- (6.5,0);
  \draw[<->] (0,-0.5) -- (0,3.5);
  
  \draw[dashed,color=nicered] (4,3) -- (4,0);
  \draw[dashed,color=nicered] (-0.5,3) -- (4,3);
  \draw[color=nicered,fill=nicered] (4,3) circle(0.7mm);
  \draw (4,2.5) node[right] {$z = x + y\text{i}$};
  \draw node at (3.8,-0.25) {$x$};
  \draw node at (-0.25,2.5) {$y$};
  
  \draw[color=neekiBlue] (0,0) -- (4,3);
  \draw[color=neekiBlue] (1.5,0.75) node[right] {$\lvert z \rvert = \sqrt{x^2 + y^2}$};
  
\end{tikzpicture}
\end{center}
\caption{The modulus, $\lvert z \rvert$, is given by Pythagoras' theorem}
\label{fig:modulus}
\end{figure}

\section{Euler's Formula}
\label{sec:CP Euler's Formula}
\begin{remember}{Euler's Formula}
Euler was this dude who had a formula:
\begin{align}
  e^{i\theta} & = \cos\theta + i\sin\theta \label{eq:eulerformula}
\end{align}
\end{remember}

\noindent In the context of the unit circle, projected onto an Argand diagram,
it starts to form an important relationship shown in figure \ref{fig:EulerFormulaOnArgand}.

\begin{figure}[!htb]
\begin{center}
\begin{tikzpicture}[scale=3]
  \draw[very thin, color=black!10] (-1.5, -1.5) grid (1.5, 1.5);
  \draw[->](0,0) -- (0,1.5)
      node[left] {$\Im$};
      
  \draw[->](0,0) -- (1.5,0)
      node[below] {$\Re$};
  
  \draw (0,0) circle(1);
  
  \draw[dashed,color=neekiBlue] (o)+(0.25,0) arc (0:60:0.25);
  \draw[->,color=neekiBlue] (o) -- ++(60:1);
  \draw[color=neekiBlue,fill=neekiBlue] ++(60:1) circle(0.03);
  \draw[color=nicered] node at (0.125,0.06125) {$\theta$};
  
  \draw[dashed,color=nicered] ++(60:1) -- (0.5,0);
  \draw[color=nicered] (0.5,0.5) node[right] {$i\sin\theta$};
  \draw[dashed,color=nicered] ++(60:1) -- (0,0.87);
  \draw[color=nicered] (0.25,0.87) node[below] {$\cos\theta$};
  
  \draw (1.1,0) node [below] {$1$};
  \draw (0,1.1) node [left] {$i$};
  
\end{tikzpicture}
\end{center}
\caption{Euler's Formula plotted on an Argand diagram}
\label{fig:EulerFormulaOnArgand}
\end{figure}