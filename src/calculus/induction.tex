\chapter{Induction}
\label{chap:I}

\begin{bigideas}{sec:I Big Ideas}
\begin{itemize}
  \item POMI establishes that a given statement is true for all $\mathbb{N}$.
  \item Start by proving (or assuming if given) that the first statement is true. This statement is called the \textbf{basis} or \textbf{base case}
  \item Next prove that \textbf{any one} statement then so is the next statement. This is known as the \textbf{inductive step}.
\end{itemize}
\end{bigideas}

\section{Set Notation}
\begin{itemize}
  \item $\mathbb{N}$ Natural numbers, $\left\{ 1,2,3,4,\ldots\right\}$
  \item $\mathbb{Z}$ Integers (from the German word for ``Number''), $\left\{\ldots,-3,-2,-1,0,1,2,3,\ldots\right\}$
  \item $\mathbb{Q}$ Quotients (fractions), $\left\{ \frac{a}{b} | a \in \mathbb{Z}, b \in \mathbb{Z}, b \neq 0 \right\}$, rationals.
  \item $\mathbb{R}$ Real numbers $\mathbb{Q} \cup \left\{\text{irrationals}\right\}$.
\end{itemize}
If a number, $A$ is a set of numbers and the number $x$ is a member of the set
$A$, then we write $x \in A$. If $x$ is not a member of $A$ then we write
$x \notin A$.

\section{Intervals}
\begin{itemize}
  \item $[a,b] = {x \in \mathbb{R} | a \leq x \leq b})$ closed interval
  \item $(a,b) = {x \in \mathbb{R} | a < x < b})$ open interval
  \item $(a,b] = {x \in \mathbb{R} | a < x \leq b})$ partly open interval
\end{itemize}
Supose that $A$ and $B$ are two sets, we say that $A$ is a subset of $B$ if $x \in A$
implies that $x \in B$ and is denoted by $A \subset B$. If $A$ is a subset of $B$
then we also say that $B$ contains the set $A$.
\begin{align}
  A \subset B \text{if no element of A can be found which is not in B} \nonumber
\end{align}
\section{Well Ordering}
Less than or equal ($\leq$) to provide an ordering on $\mathbb{N},\mathbb{Z},\mathbb{Q}$ and $\mathbb{R}$.
Given an order on a set (any set, not just those 4), we call the set
\emph{well-ordered} if every non-empty subset has a least element.
\subsection{Example}
Are any of the sets $\mathbb{N}, \mathbb{Z}, \mathbb{Q}$, and $\mathbb{R}$
well-ordered?
Ask ourselves, is $\mathbb{Z}$ well ordered? No because $\mathbb{Z} \subset \mathbb{Z}$

Is $\mathbb{N}$ well ordered? Yes, because every non-empty subset has a least
element. This is an axiomatic property of $\mathbb{N}$\footnote{That is, we define
$\mathbb{N}$ to start at $1$ (or maybe $0$).}.

\section{Propositions}
A proposition is a statement that is either true ($\true$) xor false ($\false$).

\section{The Principle of Mathematical Induction}
Mathematical induction is a technique for proving if propositions index by natural numbers
are true for all natural numbers. For example, consider the propositions $P_n$,
where $P_n$ is the satement ``$3^n$ - 1'' disivible by 2.

Supposed that $\{P_n\}$ is a collation of propositions such at
\begin{enumerate}
  \item $P_b$ is true for some fixed $b \in \mathbb{N}$.
  \item $P_n$ is true, then $P_{n+1}$ is true.
  \item The $P_n$ is true for all $n \in \mathbb{N}$.
\end{enumerate}
To use mathematical induction to prove sets of propositions like these, you
should follow the following four step algorithm:
\begin{enumerate}
  \item Prove that a proposition is $\true$ for some fixed natural number (usually
  $P_0$ or $P_1$). called the \emph{base case}.
  \item Write down your assumption about the truth of the proposition $P_k$ for
  some fixed arbitrary natural number $k$.
  \item Write down the proposition $P_{k+1}$. Using the assumption that $P_k$
  is true, prove that $P_{k+1}$ is true.
  \item State that you have proved the propositions using POMI
  \footnote{Principle of Mathematical Induction.}.
\end{enumerate}
\subsection{Example}
Show that the following formula is true for all natural numbers, $n$:
\begin{align}
  1^2 + 2^2 + 3^2 + \ldots + n^2 & = \frac{n(n+1)(2n+1)}{6} \nonumber \\
  \intertext{Firstly we must prove it for the base case}
  \text{Let $n = 1$ such that} \nonumber \\
  P_1 &= 1^2 = \frac{1\cdot(1+1)(2\cdot1 + 1)}{6} \\
  \intertext{Assume $P_k$ is true:}
   1^2 + 2^2 + 3^2 + \ldots + k^2 & = \frac{k(k+1)(2k+1)}{6} \\
  \intertext{Substitute $P_{k+1}$}
  1^2 + 2^2 + 3^2 + \ldots + (k+1)^2 & = \frac{(k+1)((k+1)+1)(2(k+1)+1)}{6} \\
    &= \frac{(k+1)(k+2)(2(k+1)+1)}{6} \\
    &= \frac{(k+1)(k+2)(2k +3)}{6} \\
    &= \frac{(k^2 +3k +2)(2k +3)}{6} \\
    &= \frac{(2k^3 +3k^2 +6k^2 +9k +4k +6)}{6} \\
    &= \frac{(2k^3 +9k^2 +13k)}{6} + 1 \\
    &= \frac{k^3}{3} + \frac{3k^2}{2} + \frac{13k}{6} + 1 \\
  1^2 + 2^2 + 3^2 + \ldots + k^2 + (k+1)^2 &= \\
    1^2 + 2^2 + 3^2 + \ldots + k^2 + k^2 + 2k + 1 &= \\
    1^2 + 2^2 + 3^2 + \ldots + 2k^2 + 2k + 1 &=\\
  \intertext{subtract 1 from both sides:}
    1^2 + 2^2 + 3^2 + \ldots + 2k^2 + 2k &= \frac{k^3}{3} + \frac{3k^2}{2} + \frac{13k}{6} \\
    1^2 + 2^2 + 3^2 + \ldots + 2k(k+1) &=
  \intertext{now i give up}
\end{align}

\begin{align}
\intertext{Proposition $P_n$:}
  1^2 + 2^2 + 3^2 + \ldots + n^2 & = \frac{n(n+1)(2n+1)}{6} \nonumber \\
\intertext{Assume $P_k$:}
  1^2 + 2^2 + 3^2 + \ldots + k^2 & = \frac{k(k+1)(2k+1)}{6} \nonumber \\
\intertext{Assume $P_{k+1}$}
  1^2 + 2^2 + 3^2 + \ldots + (k+1)^2 & = \frac{(k+1)((k+1)+1)(2(k+1)+1)}{6} \\
\intertext{Add $(k+1)^2$ to both sides} 
1^2 + 2^2 + 3^2 + \ldots + (k+1)^2 + (k+1)^2 & = \frac{(k+1)((k+1)+1)(2(k+1)+1)}{6} + (k+1)^2\\
& = \frac{(k+1)((k+1)+1)(2(k+1)+1)}{6} + \frac{6(k+1)^2}{6}\\
\end{align}
Foo
\begin{align}
  1^2 + 2^2 + \ldots + (k+1)^2 &= \frac{(k+1)((k+1)(+1))(2k+3)}{6} 
\end{align}

So since $P_1$ is true and if $P_k$ is true, then $P_{k+1}$ is true so by the
POMI $P_n$ is true for $n \in \mathbb{N}$.


\subsection{Example}
Show that $3^n -1$ is divisible by 2 for all natural numbers $n$.
\begin{align}
\intertext{Step 1: prove the base case}
  P_1 \text{says $3^1-1$ is divisble by 2 which is true} \\
  P_2 \text{says $3^2-1$ is divisible by 2 which is true} \\
\intertext{Assume $P_k$}
  3^k - 1 & \text{is divisible by 2} \\
    & \leftrightarrow 3^k - 1 = 2s | s \in \mathbb{Z} \\
\intertext{Step 3 $P_{k+1}$ says:}
  3^{k+1} - 1 &  ~\text{is divisible by 2} \\
    & \leftrightarrow 3^{k+1} - 1 = 2s' | s' \in \mathbb{Z}
\end{align} 

\subsection{Example}
Show $2^k > k^3$ for natural numbers where $n \geq 10$.
\begin{align}
  \intertext{Prove base case}
  \intertext{Assume $P_k$}
    2^k & > k^3 \\
  \intertext{$P_{k+1}$ says}
    2^{k+1} & > (k+1)^3 \\
  \intertext{Assume}
    2^k & > k^3 \\
    2^{k+1} & > 2k^3 \\
      & > k^3 + k^3 \\
      & \geq k^3 + 10k^2 \\
      & \geq k^3 + 3k^2 + 3k^2 + 4k^2 \\
      & > k^3 + 3k^2 + 3k + 1 \\
   2^{k+1} & > (k+1)^3 \\
   \intertext{Which is $P_{k+1}$} \nonumber
\end{align}

\section{Some remarks on terminology}
\textbf{Inductive arguments (ordinary English)} - An argument where we use
evidence from specific statements to conclude that something is probably.
Example: All life forms we know of depend on water to exist. Therefore, if we
discover a new life form, it will probably depend on water to exist.
\textbf{Deductive Aruments (ordinary English)} - An argument where we use
general statements to conclude that something is certain. Example: All champagne
is made in France. Dom Perignon is a champagne. Therefore Dom Perignon is made
in France.
\textbf{Mathematical Induction} is badly named, because in ordinary English it
is really a form of deduction not induction. This is because we use priniciples
to guarantee the truth of the given propositions.

\subsection{Example}
In the parlour game Nim, there are two players and two piles of matches. At each
turn a player removes some (non-zero) number of matches from one of the piles.
The player who removes the last match wins.

Prove that if the two piles contain the second number of matches at the start of
the game, then the second player can always win.

\subsubsection{Winning strategy} Player 2 takes from the opposite pile the same
number of pens that player removed.

Induction on the number of pens, $n$. This means that player 2 \emph{must} win.