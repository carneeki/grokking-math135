\chapter{Functions}
\label{chap:F}

\begin{bigideas}{sec:F Big Ideas}
\begin{itemize}
  \item Many naturally occurring quantities that vary with time can be modelled
  using \emph{functions}.
\end{itemize}
\end{bigideas}

\section{Example}
The volume of water stored in Canberra's dams is a variable that depends on
time. For any particular time $t$, we could denot the corresponding volume by
$f(t)$. In this situation
\begin{itemize}
  \item $f$ is a function,
  \item if $t$ is an input for the function, $f$, then $f(t)$ is the
  corresponding output.
\end{itemize}
Notation:
$$t \overset{f}{\to} f(t) $$

\section{Specifying a function}
A function $f$ has three parts to its definition:
\begin{itemize}
  \item The rule which explains how to get outputs form inputs,
  \item the specification of the function's \emph{domain} (ie set of inputs)
  \item the specification of the function's \emph{co-domain} (ie the set of
  where the outputs lie)
\end{itemize}
The second part of the definition is often not explicitly stated, but it is very
important. In MATH135, the third part is almost always $\mathbb{R}$.

\section{Functions as maps - a pictorial view}
% TODO: draw a surjective function
%  domain ->   co-domain
% ( . )  ->  ( . )
% ( . )  -/  ( . )
% ( . )  ->  ( . )
Function acts on \emph{every} element of the domain.
Example: $f(x) = x^2$
For example, $f(-4) = f(4) = 16$. This is OK, and is called a surjective
function.

\section{Example and terminology}
A function $f$ with domain $[0,\infty)$ is given by the rule:
$$f(x) = x^2 ~~~ \forall x \in [0,\infty] $$.
\begin{itemize}
  \item $[0,\infty)$ is the set of input values, the domain of $f$
  \item Write \texttt{Dom($[0,\infty)$}).
  \item $\forall$ means for \emph{all}.
  \item We say that $f$ is defined on $[0,\infty)$
  \item We say that $f$ is real-valued because the outputs are real numbers (the
  codomain is $\mathbb{R}$).\\
  \\
  \emph{Important note:} Technically, it would be incorrect to write ``Consider the function $f(x) = x^2$
  although this is a common abuse of notation.

  \item $f$ is the function; $f(x)$ is the value of the function at $x$.
  \item The domain of $f$ is not specified.  
\end{itemize}

$$f:[0,\infty) \to \mathbb{R}$$.
If we want to include the rule as well:
$$f:[0,\infty) \to \mathbb{R} : x \mapsto x^2$$
If we see $f:A \to B$, it means $f$ is a rule that acts on every element of $A$
and to each element of $A$ assigns a definite element of $B$.

\section{The Maximal Domain}

\section{Examples}
\begin{itemize}
  \item $f(x) = \frac{1}{x-5}$, the maximal domain: $\mathbb{R} \ {5} $.
  \item $g(x) = \sqrt{x+2}$, $x \geq -2$
\end{itemize}

\section{The range of a function}
Suppose that $f$ is a function. The \emph{range} of $f$, denoted by \texttt{Range($f$)},
is defined by:\\
\texttt{Range($f$)} $= \{f(x) \in B : x \in $ \texttt{Dom($f$)}$\}$.

Note that \texttt{Range($f$)} \textbf{is the set of all output values for $f$}.

Suppose that $f$ is defined by
$$\texttt{Dom($f$)} = \mathbb{R}, f(x) = \sin(x), \forall x \in \mathbb{R}$$

Choose codomain $\mathbb{R}$
$$f:\mathbb{R} \to \mathbb{R} : x \mapsto \sin(x)$$

Range? Is 3 an output? No.
Rnage is [-1,1]

\section{Absolute Values}
Definition, if $x \in \mathbb{R}$ then $|x|$ is defined by
$$|x| = \left\{
\begin{array}{ll}
     x & \mbox{if } x \geq 0 \\ 
    -x & \mbox{if } x < 0
\end{array}
\right.
$$
\section{More piecewise definitions}
Piecewise functions are very important in applications. Absolute value is an
example of a piecewise function.
Example:
$$
f(x) = \left\{
\begin{array}{ll}
  2 & \mbox{if } x < 0 \\
  x^2 - 1 & \mbox{if } 0 \leq x < 3 \\
  8 -2x \mbox{if } x \geq 3
\end{array}
\right.
$$

\section{Combining functions}
If two functions $f$ and $g$ have the same domain $A$, we can construct new
functions $f+g$, $f-g$ and $f\cdot g$ each with the domain $A$. These are
defined pointwise by te following formuale:
\begin{align}
  (f +g)(x) & := f(x) + g(x) \\
  (f -g)(x) & := f(x) - g(x) \\
  (f\cdot g)(x) & := f(x) \cdot g(x)
\end{align} 
We can also define $\frac{f}{g}$ by the formula:
\begin{align}
  \frac{f}{g}(x) & := \frac{f(x)}{g(x)} & \text{provided that $g(x) \neq 0$}
\end{align}

The domain of $\frac{f}{g}$ is $\{x \in A : g(x) \neq 0\}$.

\section{Composition}
If Range(f) is a subset of Dom(g) then we can define a new function, $g \circ f$
on $A$ by the rule:
%$$ (g \circ f)(x) = g(f(x)) ~~~ \forall x \in \text{\Dom{f} such that} f(x) \in \text{\Dom{g}} $$
