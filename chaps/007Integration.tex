\chapter{Integration}
\section{Big Ideas:}
First fundamental theorem of calculus:

$\int = $ accumulated rate of change of function $f$ from its rate of change
of $f'$.

Historically, Riemann (who lived in the 19$^{th}$ century) formalized calculus,
and was generalized by Lebesgue (late 19${th}$ century). The main difference
between the two is that in a Riemann construction, one creates a series of
vertical rectangles under a curve and sums their areas where as Lebesgue
constructs the rectangles and counts how many there are of particular heights.

\section{Techniques}
\begin{enumerate}
  \item Inspection
  \item Substitution
  \item Partial integration
  \item Integration of rational functions by expansion of partial fractions)
\end{enumerate}

\section{Riemann Sums}
Riemann sums follow an algorithm that works as follows:
\begin{enumerate}
  \item Subdivide the domain of the function (between points $a$ and $b$) into
  equal parts.
  \item Evaluate the function at the \emph{left} of each part, and draw a
  rectangle to the next part.
  \item Sum each rectangle.
\end{enumerate}
\begin{align}
  f(a)\left(\frac{b-a}{2}\right) + f(x_1)\left(\frac{b-a}{2}\right) & \leq \quad
    \text{area under the curve}
\end{align}
Because \emph{left hand} Riemann sums are generally too small, we can take
\emph{right hand} Riemann sums:
\begin{align}
  f(x_1)\left(\frac{b-a}{2}\right) + f(b)\left(\frac{b-a}{2}\right) & \geq \quad
    \text{area under the curve}
\end{align}
However, this will be an \emph{over estimate}. How can we improve accuracy? We
can divide the interval by increasing the number of parts. The width of each
rectangle will get thinner, but the error also decreases.
\begin{align}
  a_{n} = x < x_1 < \ldots < x_n = b
\end{align}

\noindent There is still a further optimisation to be made... Because that LH
sums are generally too small, and RH sums are generally too large:
\begin{align}
  f(x_0) \frac{b-a}{n} + \ldots + f(x_{n-1})\frac{b-a}{n}
  \quad \leq \quad A(f) \quad \leq \quad
  f(x_1) \frac{b-a}{n} + \ldots + f(x_n) \frac{b-a}{n}
\end{align}
(where $A$ is the actual area)... But, if we take all the over estimates and all
the under estimates:
\begin{align}
  m_1 \frac{b-a}{n} + \ldots +
    m_i \frac{b-a}{n} + \ldots +
      m_n \frac{b-a}{n} + \ldots
      &= \sum_{i=1}^{n} m_i \frac{b-a}{n} \\
  M_1 \frac{b-a}{n} + \ldots +
    M_i \frac{b-a}{n} + \ldots +
      M_n \frac{b-a}{n} + \ldots
      &= \sum_{i=1}^{n} M_i \frac{b-a}{n}
\end{align}
We expect:
\begin{align}
  \sum_{i=1}^{n} m_i \frac{b-a}{n}
    \leq A(f) \leq
      \sum_{i=1}^{n} M_i \frac{b-a}{n}
\end{align}
If $f$ is well behaved, and we take the limit of function:
\begin{align}
  \int_a^b f(x) dx &= \lim_{n \to \infty} \sum_{i=1}^n m_i\frac{b-a}{n} \\
                   &= \lim_{n \to \infty} \sum_{i=1}^n M_i\frac{b-a}{n}
\end{align}

\subsection{Parts of the notation:}
The $\int$ symbol is actually a letter S that has been stretched vertically. \\
$\int_a^b f(x) dx$ means we sum all the parts from $a$ to $b$ (travelling left
to right along the $x$ axis) of some function. The $dx$ means we are working
with respect to the $x$ axis.

If $a<b$, $\int_b^a f(x) dx) = - \int_a^b f(x) dx$... we have to subtract area
underneath curves... This is the \emph{signed} area.


\section{Example of Integration}
\begin{align}
\int_{0}^{1} \frac{1}{4+x^2} dx
  &= ... \\
\left(\arctan(x)\right)' &= \frac{1}{1+x^2}  \\
\left(\arctan\left(\frac{x}{2}\right)\right)' &= \frac{1}{1+(\frac{x}{2})^2}\cdot\frac{1}{2} \\
  &= \frac{2}{4+x^2} \\
\left(\frac{1}{2}\arctan\left(\frac{x}{2}\right)\right)' &= \frac{1}{4+x^2} \\
\intertext{So:}
\int_{0}^{1} \frac{1}{4+x^2} dx
  &= \frac{1}{2}\arctan\left(\frac{x}{2}\right)|_0^1 \\
  &= \frac{1}{2}\arctan(\frac{1}{2}) - \frac{1}{2}\arctan(0) \\
  &= \frac{1}{2}\arctan(\frac{1}{2})
\end{align}

\section{Properties of Integration}
Additive
$\int_a^b f(x) dx = \int_a^c f(x) dx + \int_c^b f(x) dx$

$\int_a^b (f(x)+g(x)) dx = \int_a^b f(x) dx + \int_a^b g(x) dx$

If $f$ is an even function AND $a > 0$
$\int_{-a}^{a} f(x) dx = 2\int_0^a f(x) dx$

If $f$ is an odd function AND $a > 0$
$\int_{-a}^{a} f(x) dx = 0$

Factors come out nicely
$\int_a^b k f(x) dx = k \int_a^b f(x) dx$


Squeeze theorem
If $m \leq f(x) \leq M \forall a \leq x \leq b$

$m(b-a) \leq \int_a^b f(x) dx \leq M(b-a)$

If $f(x) \leq g(x) \forall a \leq x \leq b$
then
$\int_a^b f(x) dx \leq \int_a^b g(x) dx$

Example:
1)
\begin{align}
\int_0^1 (1 + x + x^2) dx
  &= \int_0^1 (1) + \int_0^1 (x) + \int_0^1 (x^2) \\
  &= x|_0^1 + \frac{x^2}{x}|_0^1 + \frac{x^3}{3}|_0^1 \\
  &= (1-) + (\frac{1}{2}-0) + (\frac{1}{3}-0) \\
  &= 1+\frac{1}{2} +\frac{1}{3}
\end{align}
2)
\begin{align}
\int_{-\pi}^{3\pi} 2\sin(x) dx &= 2 \int_{-\pi}{3\pi}\sin(x)dx
 &= 2\left(\int_{-\pi}^{3\pi} \sin(x) dx + \int_{-\pi}^{3\pi} \sin(x) dx \right) \\
 &= 2\left(0 + (-\cos(x))|_{-\pi}^{3\pi}\right) \\
 &= 0
\end{align}
3)
\begin{align}
  \int_{-42}^{42} x^{41}e^-{x^2} dx &= 0 dx \\
  \intertext{Because odd function...}
  f(x) &= x^41 e^{x^2} \\
  f(-x) &= (-x)^41 e^(- -x^2) \\
  &= -x^41 e^{-x^2} \\
  &= -f(x)
\end{align}
4)

\begin{align}
  \int_0^\pi \sin^2(x) dx &= (\sin^3 x)' \\
    &= 3 \sin^x \cos x \\
  \intertext{know that}
  \cos(2x) &= \cos^2(x) - \sin^2(x) \\
    &= 1 - 2\sin^2(x) \\
  \intertext{hence}
  \sin^2(x) &= \frac{1 - \cos(2x)}{2} \\
  \intertext{so}
  \int_0^\pi \sin^2(x) dx &= \int_0^\pi \frac{1- \cos(2x)}{2} dx \\
    %&= \frac{1}{2} \int_0^\pi (1 - \cos(2x)) dx \\
    %&= \frac{1}{2} \left( \int_0^\pi dx - \int_0^\pi \cos(2x) dx \right) \\
    %&= \frac{1}{2} \left( \pi - \frac{1}{2}\sin ... curse you frank, don't cover the slides with hands and then move them away! \right)
    &= \frac{1}{2}
\end{align}
5)

%\begin{align}
  %\int_0^\sqrt{\pi} \sin(x^2) dx &= ...
  %\intertext{we know that }
  %\sin(x^2) \leq 1
  %\intertext{follows}
  %\int_0^\sqrt{\pi} \sin(x^2) dx &\leq \int_0^\sqrt{\pi} dx = \sqrt{\pi}
%\end{align}

%\graphturbate[Plot[Sin[x^2], {x, 0, Sqrt[Pi]}] ]


\section{Calculate average of continuous quantity}

The average value ($\overline{g}$) of a function $f$ can be calculated as
follows:
\begin{align}
  \overline{g} &= \frac{1}{b-a} \sum_{k=0}^{n-1} g(t_k) \Delta t \\
  &\to  \frac{1}{b-a} \int_a^b g(t)dt
  \intertext{so:}
  \overline{g} &= \frac{1}{b-a} \int_a^b g(t) dt
\end{align}

\subsection{Example - Calculate average temperature of cooling bar of metal}
Suppose a bar of metal is cooling from $1000\deg$ C to room temperature $20\deg$
C. The temperature is given by:
$T(t) = 20 + 980e^{-t/10}$.

This function comes from\footnote{they are not normally this pleasant}
\begin{align}
  T(0) &= 20 + 980 = 1000 \\
  \lim{t \to \infty} T(t) &= 20
\end{align}

This is the unique solution of:
\begin{align}
  \frac{dT}{dt} &= -\frac{1}{10}(T-20), T(0) = 1000 \\
\end{align}
and is an example of a first order differential equation.

If you differentiate you get
\begin{align}
  \frac{d}{dk} e^{kx} &= ke^{kx} \\
  \frac{df}{dx} &= kf
\end{align}

Calculate the average temperature of the bar after one hour (t=60):
\begin{align}
  T(t) &= 20 + 980e^{-t/10} \\
  T(60) &\approx 22.4\deg \\
\end{align}

Calculate the average temperature over the first 60 minutes:
\begin{align}
  \overline{T}
  &= \frac{1}{60} \int_0^60 T(t) dt \\
  &= \frac{1}{60} \int_0^60 (20+980e^{-t/10}) dt \\
  &= \frac{1}{60} \left(\int_0^60 20 dt + 980 \int_0^60 e^{-t/10} dt \right) \\
  &= \frac{1}{60} \left( 20t|_0^60 + 980\cdot(-10e^{-t/10})\rvert_0^60 \right) \\
  &= 20 + \frac{980}{6}(e^6 -1) \\
  &= 20 + \frac{490}{3}(1 - e^-6) \\
  &\approx 182.9 \deg C
\end{align}

Calculate volume of sphere of radius R
(note: $\int\int\int dxdyzdz = V$ but we won't need triple integration for this
example thankfully)


\section{Another section}

FFT (first fundamental theorem) of calculus requires anti-derivatives. Given
$f(x)$ with anti-derivative $F_1(x)$:, $F_1'(x) = f(x)$ on an interval, the
function:
\begin{align}
  F_2(x) &= F_1(x) + C \\
  F_2'(x) &= F_1'(x) = f(x)
\end{align}
We always have infinitely many anti-derivatives.\\
\\
Conversely, if $F_1$ and $F_2$ are anti-derivatives of $f$, then
\begin{align}
  F_1'(x) &= f(x) \\
  F_2'(x) &= f(x)
  \intertext{so}
  F_1'(x) - F_2'(x) &= 0 \\
  (F_1 - F_2)'(x) &= 0 \forall x \in (a,b) \\
  \intertext{hence}
  (F_1 - F_2) &= k \\
  F_2(x) &= F_1(x) -k \\
  \therefore (F(x) )' &= f(x) = F'(x)
\end{align}
Since the FFT requires anti-derivatives, with a new notation. The
anti-derivatives of a function $f$ are denoted by
\begin{align}
  \int f(x) dx &= F(x) + C
\end{align}
(an integral without boundaries). The $+C$ constant accounts for vertical shift
of the original function.

$\int_a^b f(x) dx :$ is the \emph{definite integral} - real numbers \\
\\
$\int f(x) dx :$ is the \emph{indefinite integral} - family of functions where
each function is given by a different $C$ constant.

There are infinitely many anti-derivatives, and it does not matter which
anti-derivative we take.

$\int_a^b f(x) dx = F_1(b) - F_1(a)$

Take $F_2$...

$F_2(x) = F_1(x) + C_2$

Hence

\begin{align}
  F_2(b) - F_2(a) &= \left(F_1(b) + C_2\right) - \left(F_1(a) + C_2\right) \\
  &= F_1(b) - F_1(a)
\end{align}

\section{Some standard integrals:}

\begin{align}
  \int x^n dx &= \frac{x^{n+1}}{n+1} + C | n \neq -1 \\
  \int \frac{1}{x} dx &= \log|x| + C | x \neq 0 \\
  \int \frac{1}{1+x^2} dx &= \arctan(x) + C \\
  \int \frac{1}{\sqrt{1-x^2}} dx &= \arcsin(x) + C \\
  \int f(x) dx &= F(x) + C \\
  \intertext{To verify, calculate $F'(x)$, should give $f(x)$. \emph{always}
  remember the integration constant!}
\end{align}

\section{Integration Techniques}
\subsection{Inspection using FFT}
\subsection{Substitution}
Uses chain rule from differentiation.

\begin{align}
  (g \circ f)'(x) &= g'(f(x))\cdot f'(x) \\
  \int g'(f(x))\cdot f'(x) dx &= \int (g \circ f)'(x) dx \\
  &= (g \circ f)(x) +C\\
  &= g(f(x)) +C
\end{align}
That is,
\begin{align}
  \int h(f(x))f'(x) dx &= \int H'(f(x))f'(x) dx \\
  &= \int (H \circ f)'(x) dx &= \\
  &= H(f(x)) +C
  \intertext{If}
  H'(x) &= h(x)
\end{align}
You only need an antiderivative of the outer function, in this case,
\begin{align}
  \int h(y) dy &= H(y) + C_1 \\
  \intertext{Let}
  y &= f(x)
  \intertext{Formally}
  \int h(f(x))f'(x) dx &= \int h(y) dy + \quad \text{use $y=f(x)$} \\
  \frac{dy}{dx} = f'(x) &= dy = f'(x) dx
\end{align}

\subsubsection{Example}
\begin{align}
  \int \frac{x}{1+x^2} dx &
  \intertext{Let}
  y &= x^2 \\
  dy &= 2x dx
  \intertext{st}
  \int \frac{x}{1+x^2} dx &= \frac{1}{2} \int \frac{dy}{1+y} \\
  &= \frac{1}{2} \log|1+y| + C \\
  &= \frac{1}{2} \log|1+x^2| +C \\
  &= \frac{1}{2} \log(1+x^2) +C
\end{align}

\subsubsection{Example 2}
\begin{align}
  \int \sin(x)\cos(x) dx &
  \intertext{Let}
  y &= \sin(x) \\
  dy &= \cos(x) dx \\
  \intertext{st}
  \int \sin(x)\cos(x) dx &= \int y dy \\
  &= \frac{1}{2} y^2 +C \\
  &= \frac{1}{2} \sin^2(x) +C \\
\end{align}
Alternatively

\begin{align}
  \int \sin(x)\cos(x) dx &= \frac{1}{2} \int \sin(2x) dx \\
  &= \frac{1}{2} \cdot \frac{1}{2} -\cos(2x) + C_2 \\
  &= -\frac{1}{4} \cos(2x) + C_2
\end{align}

%TODO WTF? how?

Remark:
\begin{align}
  \int f(y) dy & \int f(g(x)) g'(x) dx \\
  \sum f(y_i)(y_1 - y_i) &\\
  \intertext{If}
  y &= g(x) \\
  y_{i+1}-y_i = g(x+{i+1}) - g(x_i) &\approx g'(x)(x_{i+1}-x_i) \\
  \intertext{so}
  dy &= g'(x) dx
\end{align}

\subsubsection{Another example}
Anti-derivatives of $\frac{\log(x)}{x}$.
\begin{align}
  \int \frac{\log x}{x} dx &
  y &= \log{x} \\
  dy &= \frac{1}{x} dx \\
  \int \frac{\log x}{x} dx &= \int y dy \\
  &= \frac{1}{2} y^2 + C \\
  &= \frac{1}{2} (\log x)^2 +C \\
  \intertext{Check}
  (\frac{1}{2}(\log x)^2)' &= \frac{1}{2} 2 \log x \frac{1}{x} \\
  &= \log x \frac{1}{x}
\end{align}

\begin{align}
  \int \frac{1}{e^x +1} dx &
  \intertext{let}
  y &= e^x
  dy &= e^x dx
  \intertext{st}
  \int \frac{1}{e^x +1} dx &= \int \frac{1+e^x-e^x}{e^x+1} dx \\
  &= \int (1 - \frac{e^x}{e^x+1}) dx \\
  &= x - \log (e^x + 1) + C
\end{align}

\section{Substitution}
(Chain rule)
\begin{enumerate}
  \item $f$ needs to be continuous! (piecewise continuous is OK)
\begin{align}
  \int f(y(x))y'(x) dx &= \int f(y) dy \\
  y &= y(x) \\
  dy &= y'(x) dx \\
  \int f(y(x))y'(x) dx &= \int_{y(a)}^{y(b)} f(y) dy
\end{align}
  \item $y'(x)$ also continuous
  \item ($y(x)$ is continuously differentiable) \\
  $\to$ $y(x)$ is continuous \\
  $\to$ $f \circ y$ is continuous
\end{enumerate}

\subsection{Examples}
\begin{align}
  \int_0^{\pi/2} \cos^3u\sin^4u du & \\
  \intertext{Let}
  \sin u &= y \\
  \intertext{Such that}
  \cos u du &= dy \\
  \int_0^{\pi/2} \cos^3u\sin^4u du &= \int_0^{\pi/2} \cos u (1 - \sin^2 u)\sin^4u du \\
  &= \int_0^{\pi/2} \cos u \sin^4 u du - \int_0^{\pi/2}\cos u \sin^6 u du \\
  &= \int_0^{\sin(\pi/2)} y^4 dy - \int_0^{\sin(\pi/2)} y^6 dy \\
  &= \int_0^{1} y^4 dy - \int_0^{1} y^6 dy \\
  &= \frac{y^5}{5}|_0^{1} -\frac{y^7}{7}|_0^7 = \frac{1}{5} - \frac{1}{7}
  &= \frac{2}{35} \\
\end{align}

\section{By Parts}
(Product Rule)
\begin{align}
  (u(x)v(x) )' &= u'(x)v(x) + u(x)v'(x) \\
  \int_a^b (u(x)v(x))' dx &= \int_a^b u'(x)v(x) dx + \int_a^b u(x)v'(x) dx \\
  u(x)v(x)|_a^b &= \ldots
  \intertext{Usually written as}
  \int_a^b u'(x)v(x) dx &= (u(b)v(b) - u(a)v(a)) - \int_a^b u(x)v'(x) dx \\
  \intertext{sometimes}
  u &= u(x) \\
  du &= u'(x)dx \\
  \int_a^b v du &= uv|_a^b - \int_a^b u dv
\end{align}

\subsection{Example 1 - Int By Parts}
\begin{align}
  \int_0^{\pi/2} x\sin(x) dx & \\
  &= \int_0^{\pi/2} x d(-\cos x) \\
  &= -x \cos(x)|_0^{\pi/2} - \int_0^{\pi/2}(-\cos x) dx \\
  &= 0 + \int_0^{\pi/2} \cos x dx \\
  &= \Eval{\sin x}{0}{\pi/2} \\
  &= 1
\end{align}

\subsection{Example 2 - Int By Parts}
\begin{align}
  \int_0^{1} x\arctan x dx & \\
  &= \int_0^1 (\frac{1}{2}x^2)' \arctan x dx \\
  \intertext{Let}
  u(x) &= \arctan x \\
  u'(x) &= \frac{1}{1+x^2} \\
  v(x) &= \frac{1}{2}x^2 \\
  v'(x) &= x
  \intertext{Such that}
  \int_0^{1} x\arctan x dx \\
  &= \Eval{\frac{1}{2}x^2 \arctan x}{0}{1} - \int_0^1 \frac{1}{2}x^2 \frac{1}{1+x^2} dx \\
  &= \frac{1}{2}\arctan x 1 - 0 - \frac{1}{2}\int_0^1 \frac{x^2}{1+x^2} dx \\
  &= \frac{\pi}{8} -\frac{1}{2}\int_0^1 \frac{x^2}{1+x^2} dx \\
  &= \int_0^1 \frac{x^2}{1+x^2} dx \\
  &= \int_0^1 \frac{x^2 +1 -1}{1+x^2} dx \\
  &= \int_0^1 \left(1 - \frac{1}{1+x^2}\right) dx \\
  &= 1- \Eval{\arctan x}{0}{1} \\
  &= 1 - \left(\arctan1 - \arctan0\right) \\
  &= 1 - \frac{\pi}{4}
  \intertext{Hence}
  \int_0^1 x \arctan x dx &= \frac{\pi}{8} - \frac{1}{2}(1-\frac{\pi}{4}) \\
  &= \frac{\pi}{4}- \frac{1}{2}
\end{align}

\subsection{Example 3 - Int By Parts}
\begin{align}
  \int_0^1 1 \arctan x dx &
  \intertext{Let}
  u(x) &=\arctan x \\
  u'(x) &= \frac{1}{1+x^2} \\
  v(x) &= x \\
  v'(x) &= 1
  \intertext{Such that}
  \int_0^1 1 \arctan x dx &= \Eval{x \arctan x}{0}{1} - \int_0^1\frac{x}{1+x^2} dx \\
  &= \frac{\pi}{4} - \int_0^1 \frac{x}{1+x^2} dx \\
  \intertext{Use substitution, let}
  y&= x^2 \\
  dy&= 2xdx \\
  \intertext{such that}
  \frac{\pi}{4} - \int_0^1 \frac{x}{1+x^2} dx \\
  &= \frac{\pi}{4} - \frac{1}{2} \int_0^1 \frac{1}{1+y} dy \\
  &= \frac{\pi}{4} - \Eval{\frac{1}{2} \ln |1+y|}{0}{1} \\
  &= \frac{\pi}{4} - \frac{1}{2} \ln2 + \frac{1}{2} \ln 1 \\
  &= \frac{\pi}{4} - \frac{1}{2} \ln 2
\end{align}

\subsection{Example 4 - Int By Parts}
\begin{align}
  \int_1^e 1\cdot\ln x dx &
  &= \text{left as exercise}
\end{align}

\subsection{Example 5 - Int By Parts}
\begin{align}
  \int e^x \sin x dx &
  &= \int \sin(x) d(e^x) \\
  &= \sin(x)e^x - \int e^x d(\sin x) \\
  &= \sin(x)e^x - \int e^x \cos x dx \\
  &= \sin(x)e^x - \int \cos x d(e^x) \\
  &= \sin(x)e^x - \left(e^x\cos x -\int e^x d(\cos x) \right) \\
  &= (\sin(x) - \cos(x))e^x - \int \sin(x)e^x dx \\
  \intertext{Move the integral to the other side}
  2 \int e^x \sin(x) dx &= \left( \sin x - \cos x \right)e^x \\
  \int e^x \sin(x) dx &= \frac{\left( \sin x - \cos x \right)e^x}{2} + C
\end{align}

\subsection{Example 6 - Int By Parts}
\begin{align}
  \int \sin^2 x dx &
  &= \int \sin x \sin x dx \\
  &= \int \sin x d(-\cos x) \\
  &= -\sin x \cos x - \int(-\cos x) d(\sin x) \\
  &= -\sin x \cos x + \int(\cos^2x) dx \\
  &= -\sin x \cos x + \int (1-\sin^2x) dx \\
  &= x- \sin x \cos x - \int \sin^2 x dx \\
  2 \int \sin^2 x dx &= x - \sin x \cos x \\
  \int \sin^2 x dx &= \frac{x - \sin x \cos x}{2} + C \\
  &= \frac{1}{2}x - \frac{1}{4}\sin(2x) + C
\end{align}

\section{Second Fundamental Theorem of Calculus}
Let $f$ be a continuous function on $[a,b]$. The function \\
$F(x) = \int_a^x f(t) dt$ \\
for $a \leq b$. \\
$F$ is differentiable, and $F'(x) = f(x)$. \\
\ldots \\
\begin{align}
  \frac{1}{h}(F(x+h) - F(x))
  &= \frac{1}{h} \int_a^{x+h} f(t) dt - \int_a^{x} f(t) dt \\
  &= \frac{1}{h} \int_a^{x+h} f(t) dt \stackrel{h \to 0}{\to} f(x)
\end{align}

\subsection{Example 1 - SFTC}
\begin{align}
  \intertext{If}
  F(x) &= \int_1^x \ln t dt \\
  \intertext{Find}
  \frac{d}{dx} F(x^2) & \\
  F(x^2) &= \int_1^{x^2} \ln t dt \\
  \intertext{We know that $F'(x)=\ln x$}
  F(x^2) &= F(s(x)), \quad \text{with} \quad s(x) = x^2
  \intertext{No apply chain rule}
  \frac{d}{dx}(F(s(x))) &= F'(s(x))s'(x) \\
  &= F'(x^2) 2x \\
  &= \ln(x^2) 2x \\
  &= 4x \ln x
\end{align}

\subsection{Example 2 - SFTC}
\begin{align}
  \int \frac{1}{x^2 +6x +14} dx &
  \Delta &= 36 -4*14 < 0 \\
  \intertext{therefor no real roots. This means we cannot factorize this
  quadratic into two llinear factors. To handle this, we use completing the
  square.}
  \text{"completing the square"} \quad \leftrightarrow \quad 1+u^2 \\
  x^2 +6x +14 &= (x+3)^2 +5 \\
  &= 5\left(1 + \frac{x+3}{\sqrt{5}}^2\right) \\
  &= \frac{1}{5} \int \frac{1}{\frac{x+3}{\sqrt{5}}^2} dx
  &= \frac{1}{\sqrt{5}} \int\frac{1}{1+u^2} du
  \intertext{Let}
  u&= \frac{x+3}{\sqrt{5}} \\
  du &= \frac{1}{\sqrt{5}} dx \\
  \frac{1}{\sqrt{5}} \int\frac{1}{1+u^2} du
  &= \frac{1}{\sqrt{5}} \arctan u +C
\end{align}
