\chapter{Functions}

\section{Composition of Functions}

Suppose there are two functions:

\begin{align}
  g(x) &= \sqrt{x} \\
  f(x) &= -(1+x^2)
  \intertext{If we try to fit the output of $f$ into $g$:}
  g(f(x)) &= \sqrt{-(1+x^2)} \\
  \intertext{
    \begin{enumerate}
      \item This won't work for $\mathbb{R}$ (but will work for $\mathbb{C}$).
      \item It can be written as can be written as:
    \end{enumerate}
  }
  &= (g \circ f)(x)
  \intertext{You need to make sure the codomain of the inner function matches
  the domain of the outer function.}
\end{align}


Does outer and inner order matter?
\begin{align}
  f&: \mathbb{R} \to \mathbb{R}, f(x) = e^x \\
  g&: \mathbb{R} \to \mathbb{R} : x \mapsto \sin(x) \\
  (g \circ f)(x)
   &= g(f(x)) \\
   &= g(e^x) \\
   &= \sin(e^x)
\end{align}
Compare it to
\begin{align}
  (f \circ g)(x)
   &= f(g(x)) \\
   &= f(e^g(x)) \\
   &= f(e^{\sin(x)})
\end{align}
Which is very different, so yes - order matters.

\subsection{Domain and Codomain matching}
\begin{align}
  f&: \mathbb{R} \\ \{0\} \to \mathbb{R}, f(a) = \frac{1}{a} \\
  g&: \mathbb{R} \to \mathbb{R}: y \mapsto \sin(y) \\
  (g \circ f)(x)
    &= g(f(x)) \\
    &= g(\frac{1}{x}) \\
    &= \sin(\frac{1}{a}) \\
  dom(g \circ f) &= dom(f) = \mathbb{R} \\ \{ 0 \} \\
  cod(g \circ f) &= cod(g) = \mathbb{R}
  \intertext{The following function, $f_1$ could be created to extend $f$ such
  that the codomain of $f_1$ matches the domain of $g$.}
  f_1 &: \mathbb{R} \\ \{ 0 \} \to \mathbb{R} \\ \{ 0 \}, f_1(x) = \frac{1}{x}
  \intertext{Or one could extend $g$ to create $g_1$ to take a wider domain.}
\end{align}

Example:
\begin{align}
  f&: \mathbb{R} \to \mathbb{R}, f(x) = x^2 \\
  g&: \mathbb{R} \to \mathbb{R}, g(y) = \frac{1}{1+y^2} \\
  h&: \mathbb{R} \to \mathbb{R}, h(z) = \cos(z) \\
  (g \circ f)(x)
   &= g(x^2) \\
   &= \frac{1}{1+((x^2)^2)} \\
   &= \frac{1}{1+x^4} \\
  (h \circ g)(y)
   &= h(\frac{1}{1+y^2}) \\
   &= \cos(\frac{1}{1+y^2}) \\
  (h \circ (g \circ f))(x)
   &= h(\frac{1}{1+x^4}) \\
   &= \cos(\frac{1}{1+x^4}) \\
  (h \circ g(f(x)))
   &= (h \circ g)(x^2) \\
   &= \cos(\frac{1}{1+x^4})
  \intertext{Functional composition is associative}
  (h \circ (g \circ f))(x) &= ((h \circ g) \circ f)(x) \\
   &= ( h \circ g \circ f)
\end{align}

\subsection{Inverse functions}
An inverse is an 'undoing function' for a given bijective function.

\begin{align}
  (g \circ f)(x)
   &= g(f(x)) \\
   &= x
  \intertext{identity function:}
   &= id(x)
  \intertext{Likewise, $f$ can undo $g$}
  (f \circ g)(x)
   &= f(g(x)) \\
   &= x
  (g \circ f) &= id_A \quad f: A \to B \\
  (f \circ g) &= id_B \quad g: B \to A \\
\end{align}
Notation:
\begin{align}
  f^{-1}
  \intertext{Note: this does NOT necessarily mean 1/f(x).}
\end{align}

Example:
\begin{align}
  f(x) &= x^3 \\
  g(y) &= x^{\frac{1}{3}} \\
  f(g(y)) &= f(y^\frac{1}{3}) = (y^\frac{1}{3})^3 \\
  % todo write inverse
  f(x) = 2x-1
  y = 2x-1
  y+x = 2x
  \frac{1}{2}(y+1) = x
\end{align}
