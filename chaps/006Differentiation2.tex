\chapter{More Differentiation}
\section{Differentiability}
Differentiability refers to the ability for a function to be differentiated.
Suppose we have a function $f: D \to \mathbb{R}$
\begin{itemize}
  \item Critical point / stationary point
  \begin{itemize}
    \item Point in the domain where the derivative is zero, $f'(x) = 0$
    \item Or a point in the domain there the derivative does not exist.
    \item If $p$ is a critical point, $f(x)$ is called a critical or a
    stationary value.
  \end{itemize}
  \item Inflection point
  \begin{itemize}
    \item Point in the domain where the concavity changes, that is from concave
    up to concave down, or concave down to concave up.
    \item Can be characterized if $f$ is twice differentiable, then an
    inflection point, $f''(x) = 0$ if $p$ is an inflection point.
    \item The set of all inflection points is contained within the set of points
    with $f''(x) = 0$, which means not points where $f''(x) = 0$ are inflection
    points.
  \end{itemize}
\end{itemize}

\subsection{Example 1}
$  g: \mathbb{R} \to \mathbb{R}, g(x) = xe^{-x}$
Need to find critical points and inflection points:
Critical points:
\begin{align}
  g'(x) &= e^{-x} + x(-1)^{-x} \\
  &= (1-x)e^{-x} \\
  &= 0 \quad \text{iff} x=1
\end{align}

Inflection points:
\begin{align}
  g''(x) &= -e^{-x} + (1-x)(-1)e^{-x} \\
    &= (x-2)e^{-x}
  \intertext{Candidate point: $x=2$. Test if concavity changes:}
  g''(x) &< 0 \quad \text{for } \quad x < 2 \\
  g''(x) &> 0 \quad \text{for } \quad x < 2
\end{align}

If $x=2$ concavity changes so $x=2$ is indeed inflection point.

\subsection{Example 2}
$ h: \mathbb{R} \to \mathbb{R}, h(x) = |x|e^{-x} $
Need to find critical points and inflection points:
Critical points:
\begin{align}
  \intertext{for $x > 0$}
  h(x) &= xe^{-x} \\
  h'(x)
    &= (1-x)e^{-x} \\
    &= 0 \quad \text{iff} x=1
  \intertext{for $x < 0$}
  h(x) &= -xe^{-x} \\
  h'(x)
    &= -e^{-x}-x(-1)e^{-x} \\
    &= -e^{-x}(1-x) \\
    &\neq 0 \quad \text{for} x < 0
  \intertext{need to do limit test for differentiability:}
  x = 0: & \lim_{k \to 0^+} \frac{h(k)-h(0)}{k} \\
         & \lim_{k \to 0^+} \frac{ke^{-k} -0}{k} \\
         & \lim_{k \to 0^+} e^{-k} = 1
  x = 0: & \lim_{k \to 0^-} \frac{-ke^{-k}-0}{k} \\
         &= -1 \\
  \intertext{so $h'(0)$ does not exist, this is a second critical point, first
  at $x=1$ second $x=0$ where derivative does not exist}.
\end{align}

Inflection points:
\begin{align}
  x > 0: h(x) &= xe^{-x} \\
  h''(x) &=  (x-2)e^{-x} \\
  &= 0 \quad \text{iff} \quad x = 2 \\
  x > 2: h''(x) > 0 \\
  x < 2: h''(x) < 0 \\
  x < 0: h(x) &= -xe^{-x} \\
        h''(x) &= -(x-2)e^{-x} \neq 0 \\
  x = 0: \text{know} 0 < x < 2 \\
  h''(x) &= (x-2)e^{-x} \\
\end{align}
Concavity changes when cross $x=0$, so $x=0$ \emph{is} an inflection point.

Two inflection points, $x=2$ and $x=0$ second.

%TODO: include a plot of h(x) = |x|e^{-x}

\section{Local extrema}
Critical points are useful for determining the local maxima and local minima.
The first derivative test can be used to find local maxima and local minima. It
states: \\
If $f'(p) = 0$, then $f$ has a local extremum (either max or minimum) if $f'(p)$
changes sign when you cross critical point.

If $f'$ changes from positive to negative, then this corresponds with a local
maximum. \\
If $f'$ changes from negative to positive, then this corresponds with a local
minimum.

The second derivative test can be used to find local extrema. It state: \\
If $f'(p) = 0$, AND $f''(p) > 0$ then local minimum. \\
If $f'(p) = 0$, AND $f''(p) < 0$ then local maximum.

Often we are not concerned with local extrema, but sometimes we are interested
in global extrema. For this, we use the Extreme Value Theorem.

\section{Extreme Value Theorem}
Given $f: D \to \mathbb{R}$ which is continuous on $[a,b] \in D$, then $f$ has a
global maximum and global minimum for at least one point in the interval.

In other words, we know there exists points $x_1$ and $x_2$ in the interval with
$a \leq x_1 \leq b$ and $a \leq x_2 \leq b$ such that
$f(x_1) \leq f(x) \leq f(x_2)$ for all $a \leq x \leq b$.

\subsection{Example}
$f(x) = \frac{1}{x}$ on $(0, +\infty)$...

%TODO include plot of f(x) = 1/x from 0 to 100

Has no global max and min on $(0, \infty)$. Interval does not contain its
endpoints.
