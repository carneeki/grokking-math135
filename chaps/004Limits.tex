\chapter{Limits}

\section{TODO: What is a limit}
TODO

\section{TODO: uses of a limit}
TODO

\section{TODO: epsilon-delta definition of a limit}
TODO

\section{Left and Right Hand Limits}

\subsection{Right Hand Limits}
Happens when you approach a limit from the right
\begin{align}
  \lim_{x \to a^+} f(x) &= L
\end{align}

\subsection{Left Hand Limits}
Happens when you approach a limit from the left
\begin{align}
  \lim_{x \to a^-} f(x) &= L
\end{align}

\subsection{Both Limits}
By combining left and right hand limits, if they exist and are equal to each other:
\begin{align}
  \lim_{x \to a} f(x) &= L
\end{align}

\section{Other cases of interest}
\subsection{Limit does not exist}
Sometimes a function has no limit. In this case we may have to expand the
concept of limits. One such case as an example:

\begin{align}
  f(x) &= \frac{1}{x^2}
\end{align}

\noindent In a strict sense, $\lim_{x \to 0}$ does not exist as there is no
finite value such that $f(x)$ approaches a value as $x$ approaches 0. What we
can say is, for $M \in \mathbb{R}_{>0}$ we have $|f(x)| > M$ provided that
$0 < |X| < \delta$ for some $\delta \in \mathbb{R}_{>0}$. If this happens,

\begin{align}
  \lim_{x \to 0} \frac{1}{x^2} &= + \infty
\end{align}

\noindent In a general sense:
\begin{align}
  \lim_{x \to a} f_1(x) &= +\infty
  \intertext{but also...}
  \lim_{x \to a} f_2(x) &= -\infty
\end{align}

\subsection{Example: $y = \frac{1}{x}$}

TODO: include graph of y= 1/x

\begin{align}
  \lim_{x \to 0^+} \frac{1}{x} &= +\infty \\
  \lim_{x \to 0^-} \frac{1}{x} &= -\infty
\end{align}

\noindent If this is the case, the vertical line which is given by $x = a$ is
vertically asymptotic.

\subsection{Example: $\tan(x)$}

TODO: include tan(x) at $\pm$ pi/2 3pi/2 etc...


\section{Arbitrary}
Say $f(x) \to L$ when $x \to + \infty$, the notation:

\begin{align}
  \lim_{x \to +\infty} f(x) &= L
\end{align}

$|f(x) - L|$ can be made arbitrarily small provided that $x$ is sufficiently
large.

\subsection{Example: $\arctan(x)$}
TODO include arctan(x) (rotate tan(x) and rotate)

As $x$ grows sufficiently large, the narrower the $\epsilon$ gets
\begin{align}
  \lim_{x \to +\infty} \arctan(x) &= \frac{\pi}{2} \\
  \lim_{x \to -\infty} \arctan(x) &= -\frac{\pi}{2}
\end{align}

In this case, we call it a horizontal asymptote, $y = \pm \frac{\pi}{2}$
depending on whether we are moving to positive or negative infinity.

\subsection{Example:}

Provided that the limit exists, find
$\lim_{x \to +\infty} \frac{\sqrt{9x^2 +4x +5}}{x}$.

\begin{align}
  \intertext{We need to find what the limit is of the function as $x$ gets larger and larger.}
  \lim_{x \to +\infty} \frac{\sqrt{9x^2 +4x +5}}{x} & \\
  &= \lim_{x \to +\infty} \frac{\sqrt{x^2(9 +4/x +5/x}}{x} \\
  &= \lim_{x \to +\infty} \frac{|x| \sqrt{9+ 4/x +5/x}}{x} \label{eq:jump1}\\
  &= \lim_{x \to +\infty} \frac{x \sqrt{9+ 4/x +5/x}}{x} \\
  &= \lim_{x \to +\infty} \sqrt{9+ 4/x +5/x} \\
  &\approx \lim_{x \to +\infty} \sqrt{9} \\
  &\approx \lim_{x \to +\infty} 3
  \intertext{Consider the same function as $x$ gets smaller and smaller}
  \lim_{x \to -\infty} \frac{\sqrt{9x^2 +4x +5}}{x} &
  \intertext{Using the working to arrive at \ref{eq:jump1}}
  &= \lim_{x \to -\infty} \frac{|x| \sqrt{9+ 4/x +5/x}}{x} \nonumber \\
  &= \lim_{x \to -\infty} \frac{-x \sqrt{9+ 4/x +5/x}}{x} \\
  &= \lim_{x \to -\infty} -\sqrt{9+ 4/x +5/x} \\
  &\approx \lim_{x \to -\infty} -\sqrt{9} \\
  &\approx \lim_{x \to -\infty} -3
\end{align}


\section{Continuity}
Continuity is an important property of functions as it allows one to examine the
properties of a function. Sufficiently small errors in the inputs translate to
arbitrarily small errors in the output.

\subsection{Definition}
Suppose we have a function with a particular domain and real values, \\
\noindent $f: D \to \mathbb{R}, a \in D$.\\
\begin{enumerate}
  \item If $\lim_{x \to a} f(x) = f(a)$, then the function is said to be continuous
at that particular point, $x = a$.
  \item If if is continuous at all $a \in D$ then $f$ is continuous
  \item If is is said to be continuous at all $a \in D$, then $f$ is continuous.
  \item Value of f(a) is important for $f$ to be continuous at $x = a$.
  \item $f$ is continuous at $a$ iff
  \begin{enumerate}
    \item $\lim_{x \to a^+} f(x) = f(a)$ AND
    \item $\lim_{x \to a^-} f(x) = f(a)$
  \end{enumerate}
That is, $f$ is continuous iff we approach $a$ from both the left and right hand
side and we can reach $f(a)$.
\end{enumerate}

\section{Properties}
\begin{enumerate}
  \item If $f$ is continuous at $a$ and $g$ is continuous at $a$, then
  \begin{align}
    \lim_{x \to a} (f+g)(x) & \\
    &= \lim_{x \to a} ( f(x) + g(x)) \\
    &= \lim_{x \to a} f(x) + \lim_{x \to a} g(x) \\
    &= f(a) + g(a) \\
    &= (f+g)(a)
  \end{align}
  Then $f+g$ is continuous at $a$.

  \item If $f$ is continuous at $a$
  TODO: fill in from notes from Frank.

  is continuous at $a$.
  \item Similar for $\frac{f(x)}{g(x)}$ if $g(a) \neq 0$.
\end{enumerate}

\section{Library of continuous functions}
In general, a function is going to be continuous "if you can draw it without
removing the nib of your pen from the paper." Some examples:
\begin{enumerate}
  \item Polynoials
  \item Rational functions on their domain. $\frac{p_1(x)}{p_2(x)}$ if $p_2(a) \neq 0$. \\
  Dom($frac{p_1(x)}{p_2(x)}$) = $\left\{ x \in \mathbb{R} | p_2(x) \neq 0\right\}$
  \item $\sin$, $\cos$ are continuous
  \item $\arcsin$, $\arccos$, $\arctan$ are continuous
  \item Exponentials, logarithms are continuous
  \item $f: \mathbb{R} \to \mathbb{R} | f(x) = |x|$ (absolute value functions
  are continuous).
  \item $g: \mathbb{R}_{>0} \to \mathbb{R} | g(x) = \frac{1}{x}$ is continuous.
  \item $h: \mathbb{R}_{\neq 0} \to \mathbb{R} | h(x) = \frac{1}{x}$ is continuous.
  \item $h_2: \mathbb{R} \to \mathbb{R} | h_2(x) = \{ \frac{1}{x} | x \neq 0 , 0 | x = 0$ is NOT continuous
  as there is a jump from (sufficiently small value of -x) to 0 and 0 to (sufficiently small value of x).
  \item $h_3: \mathbb{R} \to \mathbb{R} | h_3(x) = \{ 1 | x > 0, -1 | x \leq 0$
  is NOT continuous near $x=0$ (jumps from -1 to 1).\footnote{This function is called the Heaviside function.}
  \item $h_4: \mathbb{R}_{x \neq 0} \to \mathbb{R} | h_4(x) = \{ 1 | x > 0, -1 x < 0$ is continuous
  (because at $x=0$ the function is not defined)
\end{enumerate}

\subsection{Examples}
These three functions (hyperbolic sin, cosine and tan functions) are continuous
\begin{align}
  \cosh(x) &= \frac{e^x + e^-x}{2} \\
  \sinh(x) &= \frac{e^x - e^-x}{2} \\
  \tanh(x) &= \frac{\sinh(x)}{\cosh{x}} = \frac{e^x - e^-x}{e^x + e^-x}
\end{align}

\begin{align}
  \text{We have seen} \quad \lim_{x \to 0} \frac{\sin(x)}{x} &= 1 \\
  \frac{\sin(-x)}{-x} &= \frac{\sin(x)}{x} 1 \\
  \intertext{$\therefore$ even function}
  f(x) &= \{ \frac{\sin(x)}{x} | x \neq 0, 1 | x = 0 \\
  f(0) &= 1 = \lim_{x \to 0} \frac{\sin(x)}{x}
  \intertext{$\therefore f$ is continuous at $x=0$}
\end{align}

\subsection{Example problem}
Find $a$ and $b$ so that the function
$f(x) = \{ -x^2 + a | x \leq 0,
        x\sin(1/x) + 1 | 0 < x \leq \frac{1}{\pi},
        bx^3 + 2 | x > \frac{1}{\pi}
$

Potentially problematic points will be near the edge cases for each piece of the
definition as each piece itself is continuous, but may not be continuous with
adjacent pieces...
\begin{enumerate}
  \item $x = 0:$ \\
   $\lim_{x \to -0} f(x) = \lim_{x \to -0}(-x^2+a) = a$ \\
   $\lim_{\to +0} f(x) = \lim_{x \to +0} (x\sin(\frac{1}{x})) +1 = 1 $ \\
  So $a = 1$
  \item $x = \frac{1}{\pi}:$ \\
  $\lim_{x \to + {1/\pi}^-} (x\sin(\frac{1}{x})) + 1 = 1$
  $\lim_{x \to {1/\pi}^+} (bx^3 +2) = b 1/\pi^3 + 2$
  To make $f$ continuous at $x = 1/\pi$ take $b$ such that $b/\pi^3 + 2 = 1$
  so b = $-\pi^3$
\end{enumerate}

Thus if $f: A\to B$ continuous at $a \in A$ and $g: B \to C$ is continuous at
$f(a) \in B$ then $g \circ f: A \to C$ continuous at $a \in A$... Or "the
composition of cfunctions is continuous".

Eg $\sin(1/x): 0 \neq x \mapsto 1/x \mapsto \sin(1/x)$

Example:
\begin{align}
  f(x) &= \sin(\frac{\tan x}{1+x^2}) \\
  & \quad \text{for} -\frac{\pi}{2} < x < \frac{\pi}{2} \\
  Dom(f) = ( -\frac{\pi}{2}, \frac{\pi}{2})
\end{align}
$\therefore f(x)$ is continuous.

\section{Consequences of Continuity}
\begin{itemize}
  \item If a function is discontinuous, \\
  \\
  $\forall \epsilon > 0 : \exists \delta > 0 : | x - a | < \delta \to | f(x) - f(a)| < \epsilon $ \\
  \\
  Meaning, for a band of errors (vertical strips representing input, $\delta$)
  you can find a band of errors (horizontal strips representing output,
  $\epsilon$).
  \item if $f: D \to \mathbb{R}$ is continuous for all points ($x$) such that
  $a \leq x \leq b (a \neq b)$, then if
  \begin{itemize}
    \item $f(a) = f(b)$ and $f(b) < c < f(a)$ then there exists at least one
    (and possibly more) $x_0$ with $a < x_0 < b$ such that $f(x_0) = c$.
    \item $f(a) < f(b)$ and $f(c) < c < f(b)$ then there exists at least one
    (and possibly more) $x_0$ with $a < x_0 < b$ such that $f(x_0) = c$.
  \end{itemize}
\end{itemize}

\subsection{Examples}
$f: \mathbb{R} \to \mathbb{R}, f(x) = e^{-x} -x$ is continuous everywhere. Show
that $e^{-x} -x$ has at least one solution.

\begin{align}
  \intertext{Start with zero, as it is easy}
  f(x) &= e^{-x} -x \\
  f(0) &= e^0 - 0 \\
       &= 1 \\
       & 1 > 0 \\
  f(1) &= e^{-1} -1 \\
       &= \frac{1}{e} -1 < 0
  \intertext{Since $f(1) < 0 < f(0)$ and $f$ is continuous on $[0,1]$, the
  intermediate value theorem (IVT) allows you to conclude that
  $f(x)= e^{-x}-x=0$ has at least one solution between $0<x<1$}
\end{align}


In tutorials there was a function...
$g(x) = 1 - \frac{x}{\sqrt{1+x^2}}$. We have to find the range of this function.

\begin{align}
  \intertext{Algebraically:}
    g(x) &= 1 - \frac{x}{\sqrt{1+x^2}} \\
         &= 1 - f(x)
  \intertext{Range of $g$ is related to range of $f$ by changing the sign, and
  then add one.}
    f(x) &= \frac{x}{\sqrt{1+x^2}}
  \intertext{For $y \in $ range of $f$, solve $y = \frac{x}{\sqrt{1+x^2}}$}
  y\sqrt{1+x^2} &= x \\
  y^2(1+x^2)    &= x^2 \\
  y^2 + y^2x^2  &= x^2 \\
  y^2 &= x^2(1-y^2) \\
  x^2 &= \frac{y^2}{1-y^2} \quad \text{need $-1<y<1$} \\
  x   &= \sqrt{y}{\sqrt{1-y^2}} \quad \forall -1 < y < 1
%TODO: complete with Frank's notes for apply check  (~35 mins)
  \intertext{As this is quite onerous, one can also use IVT...}
  f(x) &= \frac{x}{\sqrt{1+x^2}} \\
  \lim_{x \to + \infty} \frac{x}{\sqrt{1+x^2}}
    &= \lim_{x \to + \infty} \frac{x}{\sqrt{x^2+(1+\frac{1}{x^2})}} \\
    &= \lim_{x \to + \infty} \frac{x}{|x| \sqrt{1+\frac{1}{x^2}}} \\
    &= \lim_{x \to + \infty} \frac{x}{\sqrt{1+\frac{1}{x^2}}} = 1\\
  \lim_{x \to - \infty} \frac{x}{\sqrt{1+\frac{1}{x^2}}}
    &= \lim_{x \to + \infty} \frac{x}{ (-x) \sqrt{1+\frac{1}{x^2}}} = -1 \\
  \left| \frac{x}{\sqrt{1+x^2}}\right| &= \frac{|x|}{|x|\sqrt{1+x^2}}
    =  \frac{1}{\sqrt{1 + \frac{1}{x^2}}} < 1
  \intertext{$\therefore$ range of $f$ is between $-1$ and $1$, now connect back
  to $g$...}
  g(x) &= 1 - \frac{x}{\sqrt{1+x^2}} \\
  \text{so range} &= (0,2)
\end{align}


Second example, solving inequalities, $\frac{x^2+7x+2}{x-3} > 1$
\begin{align}
  \frac{x^2+7x+2}{x-3} &> 1 \\
  \frac{x^2+7x+2}{x-3} -1 &> 0 \\
  \frac{x^2+7x+2 - (x-3)}{x-3} &> 0 \\
  \frac{x^2 +6x +5}{x-3} &> 0 \\
  x &\neq 3 \\
  x^2 +6x +5 &= (x+5)(x+1) \\
  \intertext{Rational functions are continuous when defined}
\end{align}
